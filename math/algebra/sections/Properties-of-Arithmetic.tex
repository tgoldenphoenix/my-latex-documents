\chapter{Properties of Arithmetic}	% chapter 02

\section{Why Start with Arithmetic?}	% 2.1	

Algebra is the language of all advanced mathematics. Algebra gives us tools to take our concepts from arithmetic and make them general, meaning that we can use the concepts not just for arithmetic problems, but for other sorts of problems, too.

\section{Addition}

\textbf{Addition is commutative:} Let $a$ and $b$  be numbers. Then
\[ 
	a + b = b + a. 
\]

The commutative property is concerned with adding two numbers. What if we add three numbers?

\textbf{Addition is associative:} Let $a$, $b$, and $c$ be numbers. Then
\[
	(a + b) + c = a + (b + c).
\]

\textbf{WARNING!!} Students sometimes mix up the names “commutative” and “associative.” In the commutative property, the numbers are moved around (“commuted”) on the two sides of the equation. In the associative property, the numbers stay in the same place, but are grouped (“associated”) differently. 

With the commutative and associative property of addition, any addition problem can be rearranged without changing the sum. This is called the \textbf{any-order principle}

\textbf{Adding zero:} Let $a$ be a number. Then
\[
	a + 0 = a.
\]
