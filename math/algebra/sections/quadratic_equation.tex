\chapter{Quadratic equation}

\section{Introduction}

The word "quad" generally means 4 like in "quadruple". Quadratics don't have 4 of anything.

The Latin prefix \textit{quadri-} or "quad" is used to indicate the number 4, for example, quadrilateral, quadrant, quadruple, etc. However, it also very commonly used to denote objects involving the number 2. This is the case because \textit{quadratum} is the Latin word for square (hình vuông), and since the area of a square of side length $x$ is given by $x^2$, a polynomial equation having exponent two is known as a quadratic ("square-like") equation. By extension, a quadratic surface is a \textit{second}-order algebraic surface.

By analogy, since the volume of a cube of side length $x$ is $x^3$, a polynomial equation having exponent three is called a cubic equation. An equation of degree four is then unimaginatively called a quartic equation, or sometimes (more commonly in older sources) a biquadratic equation.

Quadratic functions have the parabola shape.

%$ax^2+bx+c=0$

\section{Parabola graphing}

The parabola is a conic section (a section of a cone).

\textbf{Vertex}: điểm cực đại hoặc điển cực tiểu. It is halfway between the focus and directrix.

A parabola is a curve where any point is at an equal distance from:
\begin{itemize}
	\item a fixed point (the \textbf{focus}), and
    \item a fixed straight line (the directrix)
\end{itemize}

The \textbf{axis of symmetry} goes through the focus, at right angles to the directrix.

\textbf{Roots} are also called x-intercepts or zeros.