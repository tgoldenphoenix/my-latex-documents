\chapter{Introduction to Angles}

\section{Types of Angles}

In the context of angles in trigonometry, the \textbf{initial side} is the starting ray (before rotation), and the \textbf{terminal side} is the final ray (after rotation). The initial ray is usually lined up with the positive x-axis on a Cartesian coordinate system also known as the \textbf{standard position}.

A \textbf{vertex} of an angle is the point at which the two lines or rays meet to form the angle.

Types of Angles:

\textbf{Acute angle} (Góc nhọn): $0^{\circ} < \Theta < 90^{\circ}$

\textbf{Obtuse angle} (Góc tù) $90^{\circ} < \Theta < 180^{\circ}$

\textbf{Straight angle} (Góc bẹt) $\Theta = 180^{\circ}$. Named for being a straight line.

\textbf{Reflex angle} (Góc lõm) $180^{\circ} < \Theta < 360^{\circ}$. Terminal side in Quardrants III or IV.

\textbf{Full angle} $\Theta = 360^{\circ}$

\textbf{Coterminal angles} are angles \underline{in standard position} that share the same terminal side, meaning they "end" at the same point on the coordinate plane, even though they might have different measures like the Full angle and the Zero angle. Coterminal angles are always be $360^{\circ}$ degrees apart or some whole multiple of 360.

A \textbf{quadrantal angle} is an angle in standard position whose terminal ray lies along one of the axes $0^{\circ}\, 90^{\circ}\, 180^{\circ}\, 270^{\circ}\, 360^{\circ}\,$: Right angle, straight angle, the $270^{\circ}$ angle, the full angle, the Zero angle.

\section{Angle Relationships}

\textbf{Congruent angles} are angles that have the same measure (or the same number of degrees), regardless of their orientation or the length of their sides.

\textbf{Adjacent angles} are two angles that share a common vertex (a point where two lines meet) and a common side (a line segment that both angles touch), but they do not overlap.

\textbf{Vertical angles} are the angles opposite each other when two lines cross. Vertical angles are always congruent.

\textbf{Complementary angles}: Two positive angles that add up to a right angle ($90^{\circ}$). The angles don't need to be adjacent to be complementary, they just need to sum up to a right angle.

\textbf{Supplementary angles}: Two positive angles that add up to a straight angle ($180^{\circ}$). Again, the angles don't need to be adjacent to be Supplementary. But when two supplementary angles are adjacent they form a \textbf{Linear pair}.

\textbf{Linear pair}: two adjacent positive supplementary angles.

\textbf{Transversal}: line that crosses two or more lines. Assume the two lines are parallel. We have some terms:

The transversal lines create two intersections. Each intersection has 4 angles.

\textbf{Corresponding angles}: same relative angle at each intersection. Corrsponding angles are congruent.

\textbf{Alternate interior angles}: góc so le trong, angles \underline{between} the parallel lines and on \underline{opposite} side of the transversal. Alternate interior angles are congruent.

\textbf{Alternate exterior angles}: góc so le ngoài, angles \underline{outside} the parallel lines and on \underline{opposite} side of the transversal. Alternate exterior angles are congruent.

\textbf{Interior angles of transversal}: angles \underline{between} parallel lines and on \underline{the same} side of the transversal. Interior angles of transversal are supplementary.

\textbf{Exterior angles of transversal}: angles \underline{outside} parallel lines and on \underline{the same} side of the transversal. Exterior angles of transversal are supplementary.
