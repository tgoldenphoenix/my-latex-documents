\chapter{Polynomials}

\section{Intro to Polynomials}

Polynomial = đa thức

Math expression (biểu thức) like \(3x + 5 + 2y\). \textbf{Equation} has equal sign.

Like \textbf{terms} (số hạng): has the same variable like: \(3x\ 5x\). You can add, subtract like terms.

In the expression "\(7x\)", the number 7 is the \textbf{coefficient} (hệ số), x is the variable. Biến có thể là (x, y, z,...); hệ số (a, b, c,...).

The exponents in polynomials can only be $0,1,2,3,...$ The expression $\frac{1}{x} + x^{2} + 3$ không phải là một đa thức polynomial do số hạng $1/x=x^{-1}$.

A polynomial can NOT have division by a variable. So something like $2/x$ is right out.

Monomial: an expression that has a single term.

\section{Multiplying \& Dividing polynomials}

\textbf{Long Multiplication} is a special method for multiplying larger numbers. Đây là cách \q{truyền thống} mỗi lần sẽ lùi 1 column vào trong.

When the polynomials have 3 or more terms it is easier and more reliable to use a method like Long Multiplication for Numbers.

\section{Linear Equation}

Phương trình tuyến tính, (hay còn gọi là phương trình bậc một hay phương trình bậc nhất)

\section{Quadratic equation}

The word "quad" generally means 4 like in "quadruple" but Quadratics (phương trình bậc 2) don't have 4 of anything.

The Latin prefix \textit{quadri-} or "quad" is used to indicate the number 4, for example, quadrilateral, quadrant, quadruple, etc. However, it \textit{also} very commonly used to denote objects involving the number 2. This is the case because \textit{quadratum} is the Latin word for square (hình vuông), and since the area of a square of side length $x$ is given by $x^2$, a polynomial equation having exponent two is known as a quadratic ("square-like") equation. By extension, a quadratic surface is a \textit{second}-order algebraic surface.

By analogy, since the volume of a cube of side length $x$ is $x^3$, a polynomial equation having exponent three is called a cubic equation. An equation of degree four is then unimaginatively called a quartic equation, or sometimes (more commonly in older sources) a biquadratic equation.

Quadratic functions have the parabola shape.

%$ax^2+bx+c=0$

lời giải (solution) = nghiệm (root)

Phương trình bậc 2 có dạng: 

% \ne means "not equal" to
\begin{equation}
  ax^{2}+bx+c=0\quad (a, b, c \in \mathbb{R}, \quad a \ne 0)
  \label{eq:quadratic equation}
\end{equation}

Nghiệm của Eq. (\ref{eq:quadratic equation}) là:

\[x_{1,2}=\frac{-b \pm \sqrt{b^{2}-4ac} }{2a},\quad (b^{2}-4ac \geq0)\]

Delta $\Delta=\sqrt{b^{2}-4ac}$ được gọi là một \href{https://en.wikipedia.org/wiki/Discriminant}{discriminant} (biệt thức).

