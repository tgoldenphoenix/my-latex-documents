\chapter{Quadratic equation}

\section{Introduction}

The word "quad" generally means 4 like in "quadruple" but Quadratics (phương trình bậc 2) don't have 4 of anything.

The Latin prefix \textit{quadri-} or "quad" is used to indicate the number 4, for example, quadrilateral, quadrant, quadruple, etc. However, it \textit{also} very commonly used to denote objects involving the number 2. This is the case because \textit{quadratum} is the Latin word for square (hình vuông), and since the area of a square of side length $x$ is given by $x^2$, a polynomial equation having exponent two is known as a quadratic ("square-like") equation. By extension, a quadratic surface is a \textit{second}-order algebraic surface.

By analogy, since the volume of a cube of side length $x$ is $x^3$, a polynomial equation having exponent three is called a cubic equation. An equation of degree four is then unimaginatively called a quartic equation, or sometimes (more commonly in older sources) a biquadratic equation.

Quadratic functions have the parabola shape.

%$ax^2+bx+c=0$

lời giải (solution) = nghiệm (root)

Phương trình bậc 2 có dạng: 

% \ne means "not equal" to
\begin{equation}
  ax^{2}+bx+c=0\quad (a, b, c \in \mathbb{R}, \quad a \ne 0)
  \label{eq:quadratic equation}
\end{equation}

Nghiệm của Eq. (\ref{eq:quadratic equation}) là:

\[x_{1,2}=\frac{-b \pm \sqrt{b^{2}-4ac} }{2a},\quad (b^{2}-4ac \geq0)\]

Delta được gọi là một định thức.

\section{Linear Equation}

Phương trình tuyến tính, (hay còn gọi là phương trình bậc một hay phương trình bậc nhất)

