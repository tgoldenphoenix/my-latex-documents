\chapter{Exponentiation and Logarithm}

\section{nth Root}

The "nth Root" used \textbf{n times} in a \textbf{multiplication} gives the original value

\[
  \begin{aligned}
    &\sqrt{a} \times \sqrt{a} = a\\
    &\sqrt[3]{a} \times \sqrt[3]{a} \times \sqrt[3]{a} = a
  \end{aligned}
\]

The symbol for square roots \q{\(\sqrt{}\)} is called the \textbf{radical symbol} with a little n to mean nth root.

When \(n\) is odd then \(\sqrt[n]{a^{n}}=a\)

If n is even, thì phần trong căn phải \(\geq 0\).

\vspace{10 mm}

Now let's look at some properties of nth root:

\textbf{Multiplication and Division}

\begin{equation}
  \sqrt[n]{ab}=\sqrt[n]{a} \times \sqrt[n]{b}
\end{equation}

(Note: if n is even then a and b must both be \(\geq 0\))

It also works for division:

\begin{equation}
  \begin{aligned}
    \sqrt[n]{a/b}=\sqrt[n]{a}/\sqrt[n]{b}
  \end{aligned}
\end{equation}

\[a \geq 0 \quad \text{and} \quad b > 0\]

Note that b cannot be zero, as we can't divide by zero.

\textbf{Addition and Subtraction}

nth root không có rules nào liên quan tới cộng và trừ. It also means that, unfortunately, additions and subtractions can be hard to deal with when under a root sign.

\textbf{Exponents vs Roots}

An exponent on one side of \q{=} can be turned into a root on the other side of \q{=}:

If \(a^{n}=b\) then \(a=\sqrt[n]{b}\).

Note: When n is even then b must be \(\geq 0\).

\textbf{nth Root of a-to-the-nth-Power}

\begin{tabular}{|c|c|c|}
  \hline
  & n is odd & n is even \\ \hline
  $a \geq 0$ & $\sqrt[n]{a^{n}}=a$ & $\sqrt[n]{a^{n}}=a$ \\ \hline
  $a<0$ & $\sqrt[n]{a^{n}}=a$ & $\sqrt[n]{a^{n}}=|a|$ \\ \hline
\end{tabular}

\vspace{10 mm}

\textbf{nth Root of a-to-the-mth-Power}

\[\sqrt[n]{a^{m}}=a^{\frac{m}{n}}\]

\subsection{Square Root vs Log()}

It might be best to answer with some examples.

\begin{itemize}
  \item The number $\log_7 23$ answers the question $7^{?} = 23$.
  \item The number $\sqrt[7]{23}$ answers the question $?^7 = 23$.
\end{itemize}

As you can see, these are different questions.

Some further points:

\begin{itemize}
  \item When we write $\log$, this is short for $\log_{10}$. So the number $\log 23$ answers the question $10^? = 23$.
  \item When we write $\sqrt{\phantom{x}}$, this is short for $\sqrt[2]{\phantom{x}}$. So the number $\sqrt{23}$ answers the question $?^2 = 23$.
\end{itemize}

\section{Intro to Exponentiation}

Trong expression \(8^{2}\), số 8 là base (cơ số); số 2 là exponent (số mũ or power or index/indices).

Bản chất của phép lũy thừa: start with \q{1} and then multiply or divide as many times as the exponent says

\[5^{3}= 1 \times 5 \times 5 \times 5\]


\section{Laws of Exponents}

Here are the rules (laws) of exponents:

\begin{equation}
  \begin{aligned}\setcounter{mysubequations}{0}
    &Law &Example\\
    \mysubnumber\quad &x^{1}=x &6^{1}=6\\ 
    \mysubnumber\quad &x^{0}=1 &7^{0}=1\\ 
    \mysubnumber\quad &x^{-1}=\frac{1}{x}\\ 
    \mysubnumber\quad &x^{m}x^{n}=x^{m+n} &x^{2}x^{3}=x^{2+3}=x^{5}\\ 
    \mysubnumber\quad &x^{m}/x^{n}=x^{m-n}\\ 
    \mysubnumber\quad &(x^{m})^{n}=x^{mn}\\ 
    \mysubnumber\quad &(xy)^{n}=x^{n}y^{n}\\ 
  \end{aligned}
  \label{eq:exponent_rules}
\end{equation}

\section{Fractional Exponents}

Also called "Radicals" or "Rational Exponents".

A fractional exponent like 1/n means to take the nth root: \(x^{\frac{1}{n}}=\sqrt[n]{x}\).

\section{Negative Base Exponents}

If we are working with a negative number raised to a power, the base does not include the negative part unless we use parentheses.

\begin{itemize}
  % \neq for "not equal"
  \item \(-2^{2} \neq (-2)^{2}\)
  \item \(-2^{2}=-1 \times 2^{2}=-4\)
  \item \((-2)^{2} = -2 \times -2 = 4\)
\end{itemize}

In the case of \(-2^{2} = -1 \times 2^{2}\), from the \href{https://greenemath.com/Prealgebra/16/OrderofOperationsLesson.html}{order of operations}, we know that we must perform exponent operations before we multiply.

In the case of \((-2)^{2}\), since the negative is wrapped inside of the parentheses, both are now part of the base.

\section{Negative Power Exponents}

Dividing is the inverse (opposite) of Multiplying. So a negative exponent means how many times to divide by the number.

\[a^{-n} = \frac{1}{a^{n}}\]

To change the sign (plus to minus, or minus to plus) of the exponent, use the Reciprocal (i.e. \(1/a^{n}\))

\section{Intro to Logarithm}

\begin{equation}
  \begin{split}
    \text{Cho Eq: } b^{p}&=n\\
    \text{For example: } 2^{3}&=8
  \end{split}
  \label{exponent_eqn}
\end{equation}

Với b là base (cơ số) và p là power. The exponent p is the logarithm of the number n.

\begin{equation}
  \begin{aligned}
    \log_bn&=p\\
    \log_28&=3
  \end{aligned}
  \label{log_eqn}
\end{equation}

Ta có:

\begin{equation}
  \begin{aligned}\setcounter{mysubequations}{0}
    \mysubnumber\quad &\log (1) = \log (10^{0}) = 0\\ 
    \mysubnumber\quad &\log (a\times b) = \log a + \log b\\ 
  \end{aligned}
  \label{eq:log_property}
\end{equation}

