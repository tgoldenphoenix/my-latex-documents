\chapter{Exponents, Roots and Logarithms}

\section{Exponents, Roots and Logarithms}

Exponents, Roots (such as square roots, cube roots etc) and Logarithm are all related!

Let's start with $3\times 3=9$. Using Exponents, we write it as: $3^{2}=9$. When any of those values are missing, we have a question, each with a \textbf{different notation}:

\begin{itemize}
  \item $3^{2}=?$ is the exponent question \q{what is 3 squared?}: $3^{2}=9$
  \item $?^{2}=9$ is the root question \q{what is the square root of 9?}: $\sqrt{9}=3$
  \item $3^{?}=9$ is the logarithm question \q{what is log base 3 of 9?}: $\log_3 (9)=2$
\end{itemize}

\vspace{6 mm}

Trong expression \(8^{2}\), số 8 là \textbf{base} (cơ số); số 2 là \textbf{exponent} (số mũ or power or index/indices).

Bản chất của phép lũy thừa: start with \q{1} and then multiply or divide as many times as the exponent says

\(5^{4}= 1 \times 5 \times 5 \times 5 \times 5=625\) => Giờ có 1, nhân 5 bốn lần sẽ ra bao nhiêu?

Logarithm là phép tính nghịch đảo của phép lũy thừa (exponent).

\(\log_5 625=4\) means \(625 \div 5 \div 5 \div 5 \div 5=1\) => 625 chia 5 mấy lần thì hết còn lại 1?

\vspace{6 mm}

\begin{equation*}
  \begin{split}
    \text{Cho Eq: } b^{p}&=n\\
    \text{For example: } 2^{3}&=8
  \end{split}
  \label{exponent_eqn}
\end{equation*}

Với b là base (cơ số) và p là power. The exponent p is the logarithm of the number n.

\begin{equation}
  \begin{aligned}
    \log_bn&=p\\
    \log_28&=3
  \end{aligned}
  \label{log_eqn}
\end{equation}

A logarithm answer the question: \textbf{How many of one number multiply together to make another number?}

\section{Square Root and nth Root}

The "nth Root" used \textbf{n times} in a \textbf{multiplication} gives the original value

\[
  \begin{aligned}
    &\sqrt{a} \times \sqrt{a} = a\\
    &\sqrt[3]{a} \times \sqrt[3]{a} \times \sqrt[3]{a} = a\\
    &\sqrt{2} \times \sqrt{2} = 2
  \end{aligned}
\]

A square root of $x$ is a number r whose square is $x$. For example: 3 squared is 9, so a square root of 9 is 3.

A \textbf{square root} of a number is a value that can be multiplied by itself to give the original number. It is like asking: What can we multiply by itself to get this?

A \textbf{square number} or \href{https://en.wikipedia.org/wiki/Square_number}{perfect square} (số chính phương) is an integer that is the square of an integer; in other words, it is the product of some integer with itself. For example, 9 is a square number, since it equal $3^{2}$ and can be written as $3 \times 3$.

\subsection{The Two Square Roots and the Principal Square Root}

There can be a \textbf{positive} and \textbf{negative} square root! The square root of 25 could be -5 of 5.

\[
  \begin{aligned}
    \text{Example: Solve } &x^{2}=a\\
    &\sqrt{x^{2}}=\sqrt{a}\\
    &x = \sqrt{a} \quad \text{and} \quad x = -\sqrt{a}\\
    &\text{Or: } x \pm \sqrt{a}
  \end{aligned}
\]

The two square roots is important khi ta giải phương trình tìm nghiệm. 

For example: Solve for x in $(x-3)^{2}=16$

\[
  \begin{aligned}
    \text{Start with: } &(x-3)^{2}=16\\
    \text{Square Roots: } &x-3= \pm \sqrt{16}\\
    \text{Calculate } \sqrt{16} \text{: } &x-3= \pm4\\
    \text{Add 3 to both sides: } &x=3 \pm 4\\
    \text{Answer: } &x=7 \text{ or } -1
  \end{aligned}
\]

Vậy đồ thị cắt trục Ox tại 2 điểm (có 2 nghiệm).

\vspace{.5cm}

If there are two square roots, then why do we say $\sqrt{25}=5$?

The radical symbol means \textbf{just the principal square root}. The Principal square root is sometimes called the Positive Square Root (but it can be zero). When using the radical symbol $\sqrt{25}$, we only care about the \textbf{positive (or zero) result}.

Lấy ví dụ là khi thực hiện việc lấy căn or find the square root(s) của 16 thì ta được kết quả $\pm \sqrt{16}=\pm 4$. Còn nếu chỉ xét $\sqrt{16}=4$

The symbol for square roots \q{\(\sqrt{}\)} is called the \textbf{radical symbol} with a little n to mean nth root. \textbf{Radicand} là biểu thức dưới dấu căn. The entire expression is called a \textbf{Radical Expression} (biểu thức chứa căn).

\subsection{Rules of Roots}

Now let's look at some properties of nth root and Square Roots:

\textbf{Multiplication and Division}

\begin{equation}
  \sqrt[n]{ab}=\sqrt[n]{a} \times \sqrt[n]{b}
\end{equation}

(Note: if n is even then a and b must both be \(\geq 0\))

\href{https://www.mathsisfun.com/algebra/square-root.html}{Why} does $\sqrt{xy}=\sqrt{x}\sqrt{y}$ ?

\vspace{7 mm}

It also works for division:

\begin{equation}
  \begin{aligned}
    \sqrt[n]{a/b}=\sqrt[n]{a}/\sqrt[n]{b}
  \end{aligned}
\end{equation}

\[a \geq 0 \quad \text{and} \quad b > 0\]

Note that b cannot be zero, as we can't divide by zero.

\textbf{Addition and Subtraction}

nth root không có rules nào liên quan tới cộng và trừ. It also means that, unfortunately, additions and subtractions can be hard to deal with when under a root sign.

\textbf{Exponents vs Roots}

An exponent on one side of \q{=} can be turned into a root on the other side of \q{=}:

If \(a^{n}=b\) then \(a=\sqrt[n]{b}\).

Note: When n is even then b must be \(\geq 0\).

Example: $3^{2}=9 \rightarrow 3=\sqrt{9}$

% \textbf{nth Root of a-to-the-nth-Power}

% \begin{tabular}{|c|c|c|}
%   \hline
%   & n is odd & n is even \\ \hline
%   $a \geq 0$ & $\sqrt[n]{a^{n}}=a$ & $\sqrt[n]{a^{n}}=a$ \\ \hline
%   $a<0$ & $\sqrt[n]{a^{n}}=a$ & $\sqrt[n]{a^{n}}=|a|$ \\ \hline
% \end{tabular}

\vspace{5 mm}

\textbf{nth Root of a-to-the-mth-Power}

\[\sqrt[n]{a^{m}}=a^{\frac{m}{n}}\]

The numerator becomes the exponent while the denominator is the index of the root.

Example: $\sqrt{a}=a^{\frac{1}{2}}$

\vspace{5 mm}

\textbf{Other rules that I don't know the name of}

\[\sqrt{x^{a}}= \left( \sqrt{x} \right)^{a}\]

\section{The Conjugate (biểu thức liên hợp)}

The conjugate (2 biểu thức liên hợp) is where we change the sign in the middle of two terms. We only use it in expressions with \textbf{two terms}, called "binomials".

The conjugate can be very useful because when we multiply something by its conjugate, we get \textbf{squares} like this:

\[(a+b)(a-b)=a^{2}-b^{2}\]

It can help us move a square root from the bottom of a fraction (the \textit{denominator}) to the top, or vice versa, by multiplying both top and bottom of a fraction by the conjugate of the denominator.

\q{\textbf{Rationalizing the denominator}} is when we move a root (like a square root or cube root) from the bottom of a fraction to the top.

For a fraction to be in its \q{simplest form}, the denominator should NOT be irrational. Fixing it (by making the denominator rational) is called \textbf{Rationalizing the Denominator}.

\section{Intro to Exponentiation}

\subsection{Laws of Exponents}

Here are the rules (laws) of exponents:

\begin{equation}
  \begin{aligned}\setcounter{mysubequations}{0}
    &Law &Example\\
    \mysubnumber\quad &x^{1}=x &6^{1}=6\\ 
    \mysubnumber\quad &x^{0}=1 &7^{0}=1\\ 
    \mysubnumber\quad &x^{-1}=\frac{1}{x}\\ 
    \mysubnumber\quad &x^{m} \cdot x^{n}=x^{m+n}\\ 
    \mysubnumber\quad &\frac{x^{m}}{x^{n}}=x^{m-n}\\ 
    \mysubnumber\quad &(x^{m})^{n}=x^{mn}\\ 
    \mysubnumber\quad &(xy)^{n}=x^{n}y^{n}\\ 
    \mysubnumber\quad &\sqrt[n]{x^{m}}=x^{\frac{m}{n}}\\
  \end{aligned}
  \label{eq:exponent_rules}
\end{equation}

Mấy cái dưới này tự tui suy ra:

\[
  \begin{aligned}
    &\left( \frac{x^{a}}{x^{b}} \right)^{-1} = \frac{x^{b}}{x^{a}}\\
    &x^{-2}=\frac{1}{x^{2}}\\
    &x^{-n}=\frac{1}{x^{n}}
  \end{aligned}
\]

A \textbf{Fractional Exponents} aka \q{Radicals} or \q{Rational Exponents} like 1/n means to take the nth root: \(x^{\frac{1}{n}}=\sqrt[n]{x}\).

\subsection{Negative Base Exponents}

If we are working with a negative number raised to a power, the base does not include the negative part unless we use parentheses.

\begin{itemize}
  % \neq for "not equal"
  \item \(-2^{2} \neq (-2)^{2}\)
  \item \(-2^{2}=-1 \times 2^{2}=-4\)
  \item \((-2)^{2} = -2 \times -2 = 4\)
\end{itemize}

In the case of \(-2^{2} = -1 \times 2^{2}\), from the \href{https://greenemath.com/Prealgebra/16/OrderofOperationsLesson.html}{order of operations}, we know that we must perform exponent operations before we multiply.

In the case of \((-2)^{2}\), since the negative is wrapped inside of the parentheses, both are now part of the base.

\subsection{Negative Power Exponents}

Dividing is the inverse (opposite) of Multiplying. So a negative exponent means how many times to divide by the number (starting with 1).

\[a^{-n} = \frac{1}{a^{n}}\]

To change the sign (plus to minus, or minus to plus) of the exponent, use the Reciprocal (i.e. \(1/a^{n}\))

\section{Intro to Logarithm}

The \textbf{Common Logarithms} use base 10 and is is written \textbf{without} a base, like this:

\[\log (1000)=\log_{10} (1000)=3\]

The \textbf{Natural Logarithms} use base \q{e} (Euler's Number) which is about 2.71828. For example:

\[\ln (7.389)=\log_e (7.389) \approx 2\]

\subsection{Logarithms can have Decimals}

Logarithms can have decimals for example: $\log_{10} (26)=1.41497\dotsc$

What is $\log_{10} 300=?$

\begin{align*}
  &10 \times 10=100\\
  &10 \times 10 \times 10=1000
\end{align*}

Oh no! We are either too low or too high. Muliplying \textbf{two} 10s is not enough, but multiplying \textbf{three} 10s is too many ... but what about multiplying by \textbf{two and a half} 10s? That is actually square root!

\begin{align*}
  &10\times 10 \times \sqrt{10}\\
  = &10\times 10 \times 3.16\dotsc\\
  = &316\dotsc
\end{align*}

We are close to 300, so we could say:

\[log_{10} 300 \approx 2.5\]

And using exponents we can write $300 \approx 10^{2.5}$

We can find more values (using cube roots, fourth-roots etc) but in practice it is easier to use a calculator.

\subsection{Negative Logarithms}

What is $\log_8 0.125=?$

Well, $1\div 8 =0.125$,

So $\log_8 (0.125)=-1$

Another example, $\log_5 0.008=-3$ so we have $1\div 5 \div 5 \div 5=5^{-3}$

\subsection{Properties of Logarithms}

Ta có:

\begin{equation*}
  \begin{aligned}\setcounter{mysubequations}{0}
    \mysubnumber\quad &\log_a (1) = 0\\ 
    \mysubnumber\quad &\log_a (m\times n) = \log_a m + \log_a n\\
    \mysubnumber\quad & \log_a (m/n)=\log_a m - \log_a n \\
    \text{The 3rd rule also gets us: }& \log_a (1/n)=-\log_a n \\
    \mysubnumber\quad & \log_a (m^{r})=r(\log_a m) \\
    & \log_a (a^{x})=x\\
    & a^{\log_a (x)}=x \\
  \end{aligned}
  \label{eq:log_property}
\end{equation*}

One of the powerful things about Logarithms is that they can \textbf{turn multiply into add}.

\subsection{Changing the Base}

\begin{equation}
  \log_a x=\frac{\log_b x}{\log_b a}
  \label{eq:change base of log}
\end{equation}

Another useful property is:

\[\log_a x=\frac{1}{\log_x a}\]
