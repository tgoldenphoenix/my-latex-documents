\chapter{Exponentiation and Logarithm}

\section{Intro to Exponentiation}

Trong expression \(8^{2}\), số 8 là base (cơ số); số 2 là exponent (số mũ or power or index/indices).

Bản chất của phép lũy thừa: start with \q{1} and then multiply or divide as many times as the exponent says

\[5^{3}= 1 \times 5 \times 5 \times 5\]

\subsection{Negative Base Exponents}

If we are working with a negative number raised to a power, the base does not include the negative part unless we use parentheses.

\begin{itemize}
  % \neq for "not equal"
  \item \(-2^{2} \neq (-2)^{2}\)
  \item \(-2^{2}=-1 \times 2^{2}=-4\)
  \item \((-2)^{2} = -2 \times -2 = 4\)
\end{itemize}

In the case of \(-2^{2} = -1 \times 2^{2}\), from the \href{https://greenemath.com/Prealgebra/16/OrderofOperationsLesson.html}{order of operations}, we know that we must perform exponent operations before we multiply.

In the case of \((-2)^{2}\), since the negative is wrapped inside of the parentheses, both are now part of the base.

\subsection{Negative Power Exponents}

Dividing is the inverse (opposite) of Multiplying. A negative exponent means how many times to divide by the number.

\[a^{-n} = \frac{1}{a^{n}}\]

To change the sign (plus to minus, or minus to plus) of the exponent, use the Reciprocal (i.e. \(1/a^{n}\))

\section{Intro to Logarithm}

\begin{equation}
  \begin{split}
    \text{Cho Eq: } b^{p}&=n\\
    \text{For example: } 2^{3}&=8
  \end{split}
  \label{exponent_eqn}
\end{equation}

Với b là base (cơ số) và p là power. The exponent p is the logarithm of the number n.

\begin{equation}
  \begin{aligned}
    \log_bn&=p\\
    \log_28&=3
  \end{aligned}
  \label{log_eqn}
\end{equation}

Ta có:

\begin{equation}
  \begin{aligned}\setcounter{mysubequations}{0}
    \mysubnumber\quad &\log (1) = \log (10^{0}) = 0\\ 
    \mysubnumber\quad &\log (a\times b) = \log a + \log b\\ 
  \end{aligned}
  \label{eq:log_property}
\end{equation}

