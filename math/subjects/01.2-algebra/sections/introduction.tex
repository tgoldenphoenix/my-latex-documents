\chapter{Introduction to Algebra}

\section{Basic Arithmetic terminologies}

\subsection{Addition}

kj

\subsection{subtraction}

\textbf{Subtrahend}: số bị trừ

\textbf{Minuend}: số trừ

\textbf{Difference}: hiệu

\subsection{Multiplication}

\textbf{Factor}: thừa số, nhân tử

\textbf{Product}: tích

\subsection{Division}

\textbf{Dividend}: số bị chia. Trong tài chính là cổ tức.

\textbf{Divisor}: số chia

\textbf{Quotient}: thương số

\textbf{Remainder}: phần dư, số dư

\section{Basic Algebraic terminologies}

\textbf{Algebra} (đại số) is a generalization of \textbf{arithmetic} (đại số) that introduces variables and algebraic operations other than the standard arithmetic operations, such as addition and multiplication.

\vspace{5mm}

Các nhà Toán học cũng có thói quen. Và thói quen của họ là: dùng n, m, p, q cho số nguyên, dùng x, y cho số thực và dùng z cho số phức.

Sau đây là một vài tiên đề với số thực (chấp nhận với niềm tin); với a, b, c là các số thực:

% what the `&&` mean
% https://tex.stackexchange.com/questions/159723/what-does-a-double-ampersand-mean-in-latex
\begin{equation}
    \begin{aligned}
      &\text{(1) Commutative property of addition (tính giao hoán): } &&a+b = b+a\\
      &\text{(2) tính giao hoán của phép nhân: } &&ab = ba\\
      &\text{(3) tính phân phối: } &&a(b+c) = ab + ac\\
      &\text{(4) tính đóng (closure): } &&a + b \text{, ab là số thực}\\
    \end{aligned}
    \label{key}
\end{equation}

\hl{Properties of Addition \& Subtraction}:

\begin{enumerate}
  \item Commutative property of addition (tính giao hoán): $a+b = b+a$
  \item Associative property of addition (tính kết hợp): $(a+b)+c=a+(b+c)$
  \item Identity property of addition: $0+a=a$
  \item Multiplication distributes over addition \& subtraction (tính phân phối của phép nhân): $a(b+c)=ab+ac$
  \item Subtraction is NEITHER commutative nor associative
\end{enumerate}

\hl{Properties of Multiplication \& Division}:

\begin{enumerate}
  \item Commutative of multiplication: $ab=ba$
  \item Associative of multiplication: $(ab)c=a(bc)$
  % \item Multiplying by 1: $1a=a$
  \item Division is NEITHER commutative nor associative.
  \item we can use the distributive property to divide a sum by a number, but we can’t use the distributive property to divide a number by a sum: $(a+b)\div c=a\div c + b\div c$
\end{enumerate}

Equality (đẳng thức); Inequality (bất đẳng thức)

\textbf{Interval notation} is a way of writing subsets of the real number line. For example: $(a,b),\ [a,b],\ (-\infty, c)$.

\vspace{10 mm}

Algebraic \href{https://en.wikipedia.org/wiki/Identity_(mathematics)}{identity} (hằng đẳng thức) luôn đúng với mọi số a, b:

\begin{equation}
  \begin{aligned}\setcounter{mysubequations}{0}
    \mysubnumber\quad &(a+b)^{2} = a^{2}+2ab+b^{2}\\ 
    \mysubnumber\quad &(a-b)^{2} = a^{2}-2ab+b^{2}\\ 
    \mysubnumber\quad &(a+b)^{3} = a^{3}+3ab(a+b)+b^{3}\\ 
    \mysubnumber\quad &(a-b)^{3} = a^{3}-3ab(a-b)-b^{3}\\ 
    \mysubnumber\quad &a^{2}-b^{2} = (a-b)(a+b)\\ 
    \mysubnumber\quad &a^{3}-b^{3} = (a-b)(a^{2}+ab+b^{2})\\ 
    \mysubnumber\quad &a^{3}+b^{3} = (a+b)(a^{2}-ab+b^{2})\\ 
  \end{aligned}
  \label{eq:identity}
  \end{equation}

\section{Absolute Value Equations}

\begin{align*}
  \text{Solve } |3x-7| &= 8\\
  3x-7&=8      &  3x-7&=-8\\
  x&=5         &  x&=-\frac{1}{3}
\end{align*}

\vspace{5mm}

\begin{align*}
  \text{Graph } &2|x-7 | -5 \leq -1\\
  &|x-7 | \leq 2\\
  -2 &\leq x-7 \leq 2\\
  5 &\leq x \leq 9
\end{align*}

\section{Fibonacci Sequence}

i
