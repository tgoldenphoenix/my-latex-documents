\chapter{Introduction to Algebra}

\section{Basic Algebra \& Terminologies}

\textbf{Algebra} (đại số) is a generalization of \textbf{arithmetic} (đại số) that introduces variables and algebraic operations other than the standard arithmetic operations, such as addition and multiplication.

\vspace{5mm}

Các nhà Toán học cũng có thói quen. Và thói quen của họ là: dùng n, m, p, q cho số nguyên, dùng x, y cho số thực và dùng z cho số phức.

Sau đây là một vài tiên đề với số thực (chấp nhận với niềm tin); với a, b, c là các số thực:

% what the `&&` mean
% https://tex.stackexchange.com/questions/159723/what-does-a-double-ampersand-mean-in-latex
\begin{equation}
    \begin{aligned}
      &\text{(1) tính giao hoán của phép cộng: } &&a+b = b+a\\
      &\text{(2) tính giao hoán của phép nhân: } &&ab = ba\\
      &\text{(3) tính phân phối: } &&a(b+c) = ab + ac\\
      &\text{(4) tính đóng (closure): } &&a + b \text{, ab là số thực}\\
    \end{aligned}
    \label{key}
\end{equation}

Equality (đẳng thức); Inequality (bất đẳng thức)

Interval notation is a way of writing subsets of the real number line. For example: \((a,b),\quad [a,b],\quad (-\infty, c)\)

\vspace{10 mm}

Algebraic \href{https://en.wikipedia.org/wiki/Identity_(mathematics)}{identity} (hằng đẳng thức) luôn đúng với mọi số a, b:

\begin{equation}
  \begin{aligned}\setcounter{mysubequations}{0}
    \mysubnumber\quad &(a+b)^{2} = a^{2}+2ab+b^{2}\\ 
    \mysubnumber\quad &(a-b)^{2} = a^{2}-2ab+b^{2}\\ 
    \mysubnumber\quad &(a+b)^{3} = a^{3}+3ab(a+b)+b^{3}\\ 
    \mysubnumber\quad &(a-b)^{3} = a^{3}-3ab(a-b)-b^{3}\\ 
    \mysubnumber\quad &a^{2}-b^{2} = (a-b)(a+b)\\ 
    \mysubnumber\quad &a^{3}-b^{3} = (a-b)(a^{2}+ab+b^{2})\\ 
    \mysubnumber\quad &a^{3}+b^{3} = (a+b)(a^{2}-ab+b^{2})\\ 
  \end{aligned}
  \label{eq:identity}
\end{equation}

\section{System of equations}

Hệ phương trình

% \begin{equation}
%   \begin{cases}
%     x+4y = 5\\
%     x-4y=-3
%   \end{cases} \iff
%   \begin{cases}
%     x+4y = 5\\
%     x-4y=-3
%   \end{cases} \iff
% \end{equation}

Solving by Addition or Subtraction

% `\right.` is a "phantom" right delimiter
\[
  \begin{aligned}
    &\left\{\begin{aligned} 
      x + 4y &= 5 \\ 
      x - 4y &= -3
    \end{aligned}\right. \iff 
    \left\{\begin{aligned}
      &x +4y = 5\\ 
      &2x = 2
    \end{aligned}\right.
    \\
    \iff &\left\{\begin{aligned} 
      x &= 1 \\ 
      y &= 1
    \end{aligned}\right.
  \end{aligned}
\]

Solve systems graphically

Solve with Substitution: lấy 1 phương trình dễ tìm x theo y, sau đó thay x vừa tìm vào phương trình còn lại tìm y.

\section{Absolute Value Equations}

\begin{align*}
  \text{Solve } |3x-7| &= 8\\
  3x-7&=8      &  3x-7&=-8\\
  x&=5         &  x&=-\frac{1}{3}
\end{align*}

\vspace{5mm}

\begin{align*}
  \text{Graph } &2|x-7 | -5 \leq -1\\
  &|x-7 | \leq 2\\
  -2 &\leq x-7 \leq 2\\
  5 &\leq x \leq 9
\end{align*}

\section{Fundamental Theorem of Arithmetic}

All natural numbers are either prime or can be expressed as a product of prime numbers.

\textbf{Prime Factorization:} The prime numbers when multiplied equal the number

\vspace{10 mm}

\textbf{Greatest Common Factor} (ước số chung lớn nhất): the largest positive integer that divides each of the integers.

To find GCF: list the prime factor of each number, multiply those factors that are in common.

GCF of 24 and 36:

\[
  \begin{aligned}
    24 &= 2^{3} \cdot 3\\
    36 &= 3^{2} \cdot 2^{2}\\
    \rightarrow \text{GCF} &= 2^{2} \cdot 3 = 12
  \end{aligned}\]
\[\]

\textbf{Least Common Multiple} (bội chung nhỏ nhất): the smallest positive integer that is divisible by both a and b. số nguyên dương nhỏ nhất chia hết cho cả a và b.

\section{Fibonacci Sequence}

i
