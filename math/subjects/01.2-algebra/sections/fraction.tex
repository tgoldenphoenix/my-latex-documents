
\chapter{Fraction}

\section{Introduction to Fraction}

An \textbf{Improper Fractions} is a fraction where the numerator (the top number) is greater than or equal to the denominator (the bottom number).

To get the \textbf{reciprocal} (số nghịch đảo) of a number, we divide 1 by the number.

\section{Decimal Expansionn}

We can expand a Decimal Number such as $37.29$:

\[
  \begin{aligned}
    37.29&=3 \times 10+7 \times 1+2 \times \frac{1}{10} + 9 \times \frac{1}{100}\\
    &=30+7+0.2+0.09
  \end{aligned}
\]

Sometimes the expansion goes forever:

\[\frac{1}{3}=0.33 \ldots=3 \times \frac{1}{10} + 3 \times \frac{1}{100} + 3 \times \frac{1}{1000} + \ldots\]

These are called \textbf{repeating, periodic or recurring} decimal (số thập phân vô hạn tuần hoàn), and can have a repeating pattern like this: 

\[\frac{1}{7}=0.14285142857\ldots \text{(The "142857" repeats forever)}\]

We can show the repeating pattern by putting a line over it, like this:

\[\frac{1}{7}=0.\overline{142857}\]

It is also possible for some numbers to have an endless decimal expansion \textbf{without repeating}. They are called \textbf{Irrational numbers}. An \textbf{Irrational Number} (số vô tỉ) is a real number that \textbf{cannot} be written as a simple fraction.

\vspace{10 mm}

