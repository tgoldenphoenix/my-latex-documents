\chapter{Proofing}

\section{What's a Proof}

Definition: định nghĩa.

\textbf{Proposition} (mệnh đề): a statement that is either true of false. Chúng tuy không quá quan trọng (để được xếp vào nhóm định lý) nhưng vẫn thú vị.

An \href{https://en.wikipedia.org/wiki/Axiom}{axiom}, postulate, or assumption (\textbf{tiên đề}, định đề) is a statement which we assume to be true without a proff. Sự thật hiển nhiên, luật chơi. Chúng ta không thể chứng minh tiên đề nên chấp nhận chúng.

A \textbf{Conjecture} (Sự giả định, giả sử): a statement that someone guesses to be true, although they are not yet able to prove or disprove it.

A \href{https://en.wikipedia.org/wiki/Theorem}{Theorem} (định lý): an important result that has been proven. Pythagorean Theorem

\textbf{Corollary} (hệ quả): a theorem that follows on from another theorem.

\textbf{Lemma} (bổ đề): a textit{small} result that has been proved. It is used to prove a theorem.

Proff by Induction : chứng minh bằng Phép quy nạp

contradiction: phản chứng

Deduce: suy ra

Proof: chứng minh

\section{Logical Rules}

\noindent \textbf{Modus Ponens}: If p is true and p implies q, then q is true.

Logical notation: \(p,p \rightarrow q \therefore q\)

Example: p = ``It is raining'', q = ``The ground is wet''.

\textbf{Given}: ``It is raining'' and \q{It is raining implies the ground is wet}

\textbf{Conclusion}: \q{The groud is wet}

\vspace{10mm}

\textbf{Modus Tollens}: If not q is true and p implies q, then not p is true.

\textbf{Logical notation}: \(\neg q, p \rightarrow q \therefore \neg p\)

Example: p = \q{It is raining}, q = \q{The ground is wet}.

\textbf{Given}: \q{The ground is not wet} and \q{It is raining implies the ground is wet}

\textbf{Conclusion}: \q{It is not raining}.

\vspace{10 mm}

\textbf{Hypothetical Syllogism}: If p implies q and q implies r, then p implies r.

\textbf{Logical notation}: \((p \rightarrow q), (q \rightarrow r) \therefore (p \rightarrow r)\)

\vspace{10 mm}

\textbf{Disjunctive Syllogism}: If not p is true and p or q is true, then q is true.

\textbf{Logical notation}: \(\neg p, (p \vee q) \therefore q\)
