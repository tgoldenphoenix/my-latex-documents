\chapter{Proofing}

\section{What's a Proof}

Definition: định nghĩa.

\textbf{Proposition} (mệnh đề): a statement that is either true of false. Chúng tuy không quá quan trọng (để được xếp vào nhóm định lý) nhưng vẫn thú vị.

An \href{https://en.wikipedia.org/wiki/Axiom}{axiom}, postulate, or assumption (\textbf{tiên đề}, định đề) is a statement which we assume to be true without a proff. Sự thật hiển nhiên, luật chơi. Chúng ta không thể chứng minh tiên đề nên chấp nhận chúng.

A \textbf{Conjecture} (Sự giả định, giả sử): a statement that someone guesses to be true, although they are not yet able to prove or disprove it.

A \href{https://en.wikipedia.org/wiki/Theorem}{Theorem} (định lý): an important result that has been proven. Pythagorean Theorem

\textbf{Corollary} (hệ quả): a theorem that follows on from another theorem.

\textbf{Lemma} (bổ đề): a textit{small} result that has been proved. It is used to prove a theorem.

Proff by Induction : chứng minh bằng Phép quy nạp

contradiction: phản chứng

Deduce: suy ra

Proof: chứng minh

\section{Logical Rules}

Chấp nhận bằng niềm tin.

\noindent \textbf{Modus Ponens}: If p is true and p implies q, then q is true.

Logical notation: \(p,p \rightarrow q \therefore q\)

Example: p = ``It is raining'', q = ``The ground is wet''.

\textbf{Given}: ``It is raining'' and \q{It is raining implies the ground is wet}

\textbf{Conclusion}: \q{The groud is wet}

\vspace{10mm}

\textbf{Modus Tollens}: If not q is true and p implies q, then not p is true.

\textbf{Logical notation}: \(\neg q, p \rightarrow q \therefore \neg p\)

Example: p = \q{It is raining}, q = \q{The ground is wet}.

\textbf{Given}: \q{The ground is not wet} and \q{It is raining implies the ground is wet}

\textbf{Conclusion}: \q{It is not raining}.

\vspace{10 mm}

\textbf{Hypothetical Syllogism}: If p implies q and q implies r, then p implies r.

\textbf{Logical notation}: \((p \rightarrow q), (q \rightarrow r) \therefore (p \rightarrow r)\)

\vspace{10 mm}

\textbf{Disjunctive Syllogism}: If not p is true and p or q is true, then q is true.

\textbf{Logical notation}: \(\neg p, (p \vee q) \therefore q\)

\vspace{10 mm}

\textbf{Addition}: If p is true, then p or q is true.

\textbf{Logical notation}: \(p \therefore (p \vee q)\)

\vspace{10 mm}

\textbf{Simplification}: If p and q are true, then p is true.

\textbf{Logical notation}: \((p \wedge q) \therefore p\)

\vspace{10 mm}

\textbf{Conjunction}: If p is true and q is true, then p and q are true.

\textbf{Logical notation}: \(p,q \therefore (p \wedge q)\)

\section{Mathematical Sets}

A \href{https://en.wikipedia.org/wiki/Rational_number}{rational number} (số hữu tỉ) viết dưới dạng p/q trong đó p và q là số nguyên và q khác không; p is numerator (tử số) và q là denominator (mẫu số).

A \href{https://en.wikipedia.org/wiki/Real_number}{Real number} (số thực) bao gồm tất cả các số hữu tỉ, như số nguyên (5) và phân số \(\frac{2}{3}\), và tất cả các số vô tỉ, như \(\sqrt{2}, \pi\), v.v.

These are the common sets:

\begin{itemize}
  \item $\mathbb{N}$ --- The Natural Numbers (Counting Numbers): $\{1,2,3,...\}$
  \item $\mathbb{W}$ --- The Whole Numbers: $\{0,1,2,3,...\}$
  \item $\mathbb{Z}$ --- The Integers: $\{...,-3,-2,-1,0,1,2,3,...\}$
  \item $\mathbb{Q}$ --- The Rational Numbers: $\left\{ \frac{p}{q} \mid p,q \in \mathbb{Z}, q \neq 0 \right\}$
  \item $\mathbb{I}$ (commonly written as $\mathbb{R} \setminus \mathbb{Q}$) --- The Irrational Numbers: $\{x \mid x \not\in \mathbb{Q}, x \in \mathbb{R}\}$
  \item $\mathbb{R}$ --- The Real Numbers: $\{x \mid x \text{ can be written as a decimal}\} = \{x \mid x \in \mathbb{Q} \text{ OR } x \in \mathbb{I}\}$
  \item $\mathbb{C}$ --- The Complex (Imaginary) Numbers: $\{a+bi \mid a,b \in \mathbb{R} , i=\sqrt{-1}\}$
\end{itemize}

Some other common sets are:

\begin{itemize}
  \item $\mathbb{U}$ --- The Universal Set: \{All objects currently under discussion\}
\end{itemize}

The \textbf{Cardinality} of a set is the number of distinct elements in the set. If the cardinality of a set is a whole number, we call it a \textbf{finite set}. Otherwise, we call it an \textbf{infinite set}.

\[|A| \text{ is the cardinality of set A}\]

A is a \textbf{Subset} of B, $A \subseteq B$, if every element of A is also an element of B.

\[(a\in A \Longrightarrow a\in B)\]

The \textbf{Power Set of A}, P(A), is the set of all possible subsets of A.

The \textbf{Complement}, \(A^\complement \text{ or } A^\prime\), of a set A is the set of all elements in the universal set that are NOT elements of A.

\[A^\complement = \{x \mid x \in \mathbb{U} \quad \text{and} \quad x \not\in A \}\]

The \textbf{Union} (hợp, hội) of two sets, $A \cup B$ is the set containing all the elements from either A or B.

\[A \cup B = \{ x \mid x \in A \text{ OR } x \in B \}\]

The \textbf{Intersection} (giao) of two sets, $A \cap B$, is the set of elements in both sets A and B.

\[a \cap B = \{ x \mid x\in A \text{ AND } x\in B \}\]

\section{Quantifiers}

\q{For all, for every} is the \textbf{Universal Quantifier} $\forall$

\q{There exists} is the \textbf{Existential Quantifier} $\exists$

\textbf{Negations:}

\[
\begin{aligned}
  &\neg(A \text{ or } B) = \neg A \text{ and } \neg B\\
  &\neg(A \text{ and } B)= \neg A \text{ or } \neg B\\
\end{aligned}
\]


\section{Direct Proofs}

\section{Contrapositive}

Trái ngược, tương phản

\[p \rightarrow q \equiv \neg q \rightarrow \neg p\]

\q{If and Only If} or (iff) is the \textbf{Two Way Proff}:

\[p \implies q \quad q \Longrightarrow p\]

\[p \iff q\]

\section{Proof by Contradiction}

Assume the opposite of what we want to prove, then show a contradiction.

\section{Divisibility}

In mathematics, \href{https://en.wikipedia.org/wiki/Parity_%28mathematics%29}{parity} (tính chẳn lẻ) is the property of an \textbf{integer} of whether it is even or odd.

The above definition of parity applies ONLY to integer numbers, hence it cannot be applied to numbers with decimals or fractions like 1/2 or 4.6978. See the section "Higher mathematics" below for some extensions of the notion of parity to a larger class of "numbers" or in other more general settings. 

An even number is an integer of the form

\[x=2k,\ (k \in \mathbb{Z})\]

The phrase \q{2 divides x} means that x is divisible by 2 — in other words, when you divide x by 2, the remainder is 0.

\[2 \mid x \text{ means } \exists k \in \mathbb{Z} \text{ such that } x = 2k\]
