\chapter{Linear Algebra}

\section{Introduction}

Linear Algebra (đại số tuyến tính)

Vectors viết bằng square bracket $\vec{v}=\begin{bmatrix}
a\\b\\c
\end{bmatrix}$ hay round bracket $\vec{v}=\begin{pmatrix}
a\\b\\c
\end{pmatrix}$ đều được.

Vectors chỉ có hướng và magnitude, không có position. Tức là có thể drag vector đi đến chỗ khác on the coordinate system nhưng vẫn không làm thay đổi vector.

\section{Type of operations on vectors}

A \textbf{Scalar operations} happens between a vector and a normal number such as: $\vec{v} \times x$ and $\vec{v} \div x$. According to the Oxford dictionary, "scalar" is a quantity that has size but no direction. In this case, the number $x$ is the scalar and the magnitude of the vector $\vec{v}$ is \textit{scaled} (multiply or divided) by another factor (which is $x$) but the direction of the scaled vector vẫn remain trên một đường thẳng (có thể ngược chiều nếu multiply với negative numbers).

\textbf{Vectors operations} happens between two vectors. This could be either an addition or a subtraction.

There are two ways to multiply vectors: dot product and cross product. \textbf{Dot product} (tích vô hướng) $\vec{a} \cdot \vec{b}$ returns a real number (a scalar), không trả về vector: $$\vec{a} \cdot \vec{b} = a_{x}b_{x} + a_{y}b_{y} = \| \vec{a} \| \| \vec{b} \| \cos\Theta$$

Cross product (tích vector):

The \textbf{length} (or magnitude, norm, size) of a vector $\vec{a}=\begin{bmatrix}
a_x\\
a_y\\
\vdots\\
a_n
\end{bmatrix}$ is calculated as: $$\| \vec{a} \| =\sqrt{a_x^2 + a_y^2 + \dots + a_n^2}$$

This definition of vector length comes from the \textit{Pythagorean theorem}. Note that the notation of double "pipe" is different from the notation for \textit{absolute values} $|v|$.

A relationship between vector length and dot product: $$\|\vec{a}\| ^2=\vec{a}\cdot\vec{a}$$

To know the vector between two points $A(x_1,y_1,z_1)$ and $B(x_2,y_2,z_2)$ in \LaTeX: $$\overrightarrow{AB} = (x_2 - x_1, y_2 - y_1, z_2 - z_1)$$

The \textbf{Unit Vector} (or Normalized Vector) is a vector with length of 1, $\|\widehat{v}\| = 1$. Take any arbitrary vector divided by its length, the result is a vector that points in the same direction as the original vector but that has a length of 1: $$\widehat{u} = \frac{\vec{v}}{\|\vec{v}\|}$$

Some popular unit vector include:$$
\widehat{e_x}=\begin{bmatrix}
1\\0\\0
\end{bmatrix}
\quad
\widehat{e_y}=\begin{bmatrix}
0\\1\\0
\end{bmatrix}
\quad
\widehat{e_z}=\begin{bmatrix}
0\\0\\1
\end{bmatrix}
$$
