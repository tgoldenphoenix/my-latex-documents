\chapter{Graphing Functions and Inequalities}

\section{Introduction}

In mathematics, \href{https://en.wikipedia.org/wiki/Analytic_geometry}{analytic geometry}, also known as coordinate geometry or Cartesian geometry (hình học giải tích, hình học tọa độ), is the study of geometry using a coordinate system. This contrasts with synthetic geometry. 

\section{Linear equation}

\textbf{Q:} Create equation of a line from 2 points \(A(x_{1},y_{1})\text{ and } B(x_{2},y_{2})\).

\textbf{Slope} (hệ số góc) cho biết rate of change. Positive \textbf{rate of change} means the line goes uphill:

\begin{equation}
  m = \frac{y_{2}-y_{1}}{x_{2}-x_{1}}
  \label{eq:3.1}
\end{equation}

2 parallel lines have \textit{the same} slope.

Find the \textbf{y-intercept} by using Eq: \(y = mx + b\), với slope m vừa tìm được, giải tìm b. Y-intercept là điểm giao với trục Oy có giá trị x = 0.

\par\rule{\textwidth}{0.5pt}

\textbf{Q:} Graph this line \(3x-4y=-6\)

Find the \textbf{x-intercept} by making y = 0. Then find the \textbf{y-intercept} by making x = 0.

\vspace{5mm}

\textbf{Q:} Graph this inequality \(y < -\frac{1}{3}x+2\)

Tìm tọa độ 2 điểm; vẽ đường thẳng.

\begin{itemize}
  \item Nếu \(\le\) thì vẽ straight line and shade under the line.
  \item Nếu > thì vẽ dashed line and shade above the line.
\end{itemize}

\textbf{Q: }Graphing Systems of Inequalities

\vspace{5mm}

Tính \textbf{Distance between 2 points} bằng Eq \ref{eq:3.2} below:

\begin{equation}
  d = \sqrt{(x_{1}-x_{2})^{2} + (y_{1}-y_{2})^{2}}
  \label{eq:3.2}
\end{equation}

\vspace{5mm}

\section{Parabola graphing}

The parabola is a conic section (a section of a cone).

\textbf{Vertex}: điểm cực đại hoặc điển cực tiểu. It is halfway between the focus and directrix.

A parabola is a curve where any point is at an equal distance from:
\begin{itemize}
	\item a fixed point (the \textbf{focus}), and
    \item a fixed straight line (the directrix)
\end{itemize}

The \textbf{axis of symmetry} goes through the focus, at right angles to the directrix.

\textbf{Roots} are also called x-intercepts or zeros.

