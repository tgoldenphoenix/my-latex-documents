\chapter{Graphing Functions and Inequalities}

\section{Introduction and Basic Terms}

In mathematics, \href{https://en.wikipedia.org/wiki/Analytic_geometry}{analytic geometry}, also known as coordinate geometry or Cartesian geometry (hình học giải tích, hình học tọa độ), is the study of geometry using a coordinate system. This contrasts with synthetic geometry. 

The \textbf{domain} is the set of all possible x-values which will make the function "work", and will output real y-values.

The \textbf{range} is the resulting y-values we get after substituting all the possible x-values.

\section{Linear equation}

\textbf{Q:} Create equation of a line from 2 points \(A(x_{1},y_{1})\text{ and } B(x_{2},y_{2})\).

\textbf{Slope} (hệ số góc) cho biết rate of change. Positive \textbf{rate of change} means the line goes uphill:

\begin{equation}
  m = \frac{y_{2}-y_{1}}{x_{2}-x_{1}}
  \label{eq:3.1}
\end{equation}

2 parallel lines have \textit{the same} slope.

Find the \textbf{y-intercept} by using Eq: \(y = mx + b\), với slope m vừa tìm được, giải tìm b. Y-intercept là điểm giao với trục Oy có giá trị x = 0.

\par\rule{\textwidth}{0.5pt}

\textbf{Q:} Graph this line \(3x-4y=-6\)

Find the \textbf{x-intercept} by making y = 0. Then find the \textbf{y-intercept} by making x = 0.

\vspace{5mm}

\textbf{Q:} Graph this inequality \(y < -\frac{1}{3}x+2\)

Tìm tọa độ 2 điểm; vẽ đường thẳng.

\begin{itemize}
  \item Nếu \(\le\) thì vẽ straight line and shade under the line.
  \item Nếu > thì vẽ dashed line and shade above the line.
\end{itemize}

\textbf{Q: }Graphing Systems of Inequalities

\vspace{5mm}

Tính \textbf{Distance between 2 points} bằng Eq \ref{eq:3.2} below:

\begin{equation}
  d = \sqrt{(x_{1}-x_{2})^{2} + (y_{1}-y_{2})^{2}}
  \label{eq:3.2}
\end{equation}

\vspace{5mm}

\section{Parabola graphing}

The parabola is a conic section (a section of a cone).

\textbf{Vertex}: điểm cực đại hoặc điển cực tiểu. It is halfway between the focus and directrix.

A parabola is a curve where any point is at an equal distance from:
\begin{itemize}
	\item a fixed point (the \textbf{focus}), and
    \item a fixed straight line (the directrix)
\end{itemize}

The \textbf{axis of symmetry} goes through the focus, at right angles to the directrix.

\textbf{Roots} are also called x-intercepts or zeros.

\section{Graphing Exponential Functions}

An exponential function of the form \(f(x) = b^{x}, b > 0, b \neq 1\), has the following characteristics:

\begin{itemize}
  \item Horizontal asymptote (tiệm cận ngang): \(y=0\)
  \item Domain (x-values): \((-\infty, \infty)\)
  \item Range (result y values): \((0, \infty)\)
  \item Increasing if \(b>1\)
  \item Decreasing if \(0<b<1\)
\end{itemize}

\vspace{10 mm}

\textbf{Q:} Why can’t the base of the exponential function be less than zero?

Nên nhớ mình đang học \href{https://en.wikipedia.org/wiki/Real_analysis}{Real analysis} (giải tích thực).

Now consider the function \(f(x) = (-4)^{x}\).

For integer values of x, our function is well defined.

The real problem arises when we extend x to the entire real number line. For example, if we let x=1/2, we have \((-4)^{\frac{1}{2}}=\sqrt{-4}=2i\). Within the framework of complex analysis, where the entire complex field is our playing ground, such a result would be perfectly acceptable. In Real Analysis, we restrict ourselves to the real number line.

\section{Graphing Logarithmic Functions}

Every logarithmic function is the inverse of an exponential function.

For any real number $x$ and constant \(b>0, b\neq 1\), we can see the following characteristics in the graph of \(f(x)=\log_b(x)\):

\begin{itemize}
  \item one-to-one function
  \item vertical asymptote (tiệm cận đứng): \(x=0\)
  \item key points: x-intercept: (1,0), (b,1) and \((\frac{1}{b}, -1)\)
  \item y-intercept: none
  \item domain (x-values): \((0, \infty)\)
  \item Range (result y values): \((-\infty, \infty)\)
  \item increasing if \(b>1\)
  \item decreasing if \(0<b<1\)
\end{itemize}

