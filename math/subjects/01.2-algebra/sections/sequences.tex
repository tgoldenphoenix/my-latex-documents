\chapter{Sequences}

\section{Sequences}

A \textbf{sequence} (dãy số) is a list of things (usually numbers) that are in order.

A sequence is like a \textbf{Set}, except:

\begin{itemize}
  \item the terms are \textbf{in order} (with Sets the order does not matter)
  \item the same value can appear many times (no duplicate in Sets)
\end{itemize}

\section{Special Sequences}

\subsection{Arithmetic Sequences}

An \textbf{Arithmetic Sequence} (or Arithmetic progression) is made by \textbf{adding} the same value each time. The value added each time is called the \textbf{common difference} (d). The common difference could also be negative.

\[\{a,\ a+d,\ a+2d,\ a+3d,\ \ldots\}\]

And we can make the rule:

\[x_{n}=a+d(n-1)\]

Where:

\begin{itemize}
  \item a is the first term
  \item d is the common difference
\end{itemize}

We use \q{n-1} because \textbf{d} is not used in the 1st term

\subsection{Geometric Sequences}

A \textbf{Geometric Sequence} is made by \textbf{multiplying} by the same value each time. What we multiply by each time is called the \textbf{common ratio}. The common ratio can be less than 1 but not zero.

\subsection{Triangular Numbers}

The \textbf{Triangular Number Sequence} is generated from a pattern of dots which form a triangle. By adding another row of dots and counting all the dots we can find the next number of the sequence.

But it is easier to use this rule:

\[x_{n}=n(n+1)/2\]

\section{Partial Sums}

When we \textbf{sum} up just \textbf{part} of a sequence it is called a \textbf{Partial Sum} (tổng riêng).

But a \textbf{sum} of an \textbf{infinite} sequence is called an \textbf{Infinite Series} (chuỗi) (it sounds like another name for sequence, but it is actually a sum).

Partial Sums are sometimes called \textbf{Finite Series}.

Partial Sums are often written using the symbol Sigma $\sum$ meaning \q{add them all up or sum up}.

For example:

\begin{align*}
  &\sum\limits_{n=1}^4 n=1+2+3+4=10\\
  &\sum\limits_{n=1}^4 n^{2}=1^{2}+2^{2}+3^{2}+4^{2}=30\\
  &\sum\limits_{n=1}^4 (2n+1)=3+5+7+9=24
\end{align*}

\subsection{Properties}

\textbf{Multiplying by a Constant Property}

Say we have something we want to sum up, let's call it $a_{k}$. $a_{k}$ could be $k^{2}$, or $k(k-7)+2$, or ... anything really.

And \textbf{c} is some constant value (like \textbf{2} or \textbf{-9.1}, etc), then:

\[\sum\limits_{k=m}^n ca_{k}=c\sum\limits_{k=m}^n a_{k}\]

In other words: if every term

\section{Series}

2

\section{Sum Of Numbers From 1 To N}

\begin{equation}
  S_{\text{1 to N}}=\frac{n(n+1)}{2}
\end{equation}

Let A represents the Sum from 1 to n

\begin{equation*}
% aligned must be inside math mode
% No line number
% \begin{align} do NOT need to be inside math mode
\begin{aligned}
  A&=1 &+ &2 &+ &3 &+ &\cdots &+ &(n-1) &+ &n\\
  A&=n &+ &n-1 &+ &n-2 &+ &\cdots &+ &2 &+ &1\\
  \midrule
  2A&=(n+1) &+ &(n+1) &+ &(n+1) &+ &\cdots  &+ &(n+1) &+ &(n+1)\\
  2A&= n(n+1)\\
\end{aligned}
\end{equation*}

