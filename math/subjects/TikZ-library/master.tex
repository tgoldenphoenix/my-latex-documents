\documentclass{article}
\usepackage[utf8]{inputenc}

% === Dùng tiếng Việt
\usepackage[vietnamese=nohyphenation]{hyphsubst} % Để nó không hiện warning
\usepackage[vietnamese]{babel}

\usepackage{tikz, pgfplots}
\usetikzlibrary{positioning}

\title{Learn Tikz}
\author{Dr. Trefor Bazett}
\date{March 2022}

\begin{document}

\maketitle
%\tableofcontents

\section{Notes}

Nhớ semi-colon khi xài Tikz này.

\section{Playgrounds}

%\vspace{1in}

\begin{figure}
  % try deleting [->]
	\tikz \draw[->] (0,0) -- (2,1) -- (3,2);
	
	\caption{Draw a line}
\end{figure}

\begin{figure}
	\begin{tikzpicture}
		\draw (0,0) circle (1);
		\draw (2,0) circle (1.5in);
		\draw (5,0) ellipse (10pt and 20 pt);
		
		\draw node at (3,0) {$f(x)$};

    % The circle is attached to the west of the node (text)
    % try anchor=east
    \filldraw (6,0) circle (0.1cm) node[anchor=west]{Anchored Node};
	\end{tikzpicture}
	
	\caption{Circles, ellipse}
\end{figure}

\begin{figure}
	\begin{tikzpicture}
		\draw (0,0) rectangle (5,4);
		
		% Try commenting this grid line
		% By default, grid lấy unit là 1
		\draw (0,0) grid (5,4);
	\end{tikzpicture}
	
	\caption{Rectangle}
\end{figure}

\newpage

\begin{figure}
\begin{center}
  % What inside the [] is the parameter
	\begin{tikzpicture}[transform canvas={scale=4.0}]  % [scale=4] sẽ khác, try it!
	
	  \draw[blue] (0,1) arc (90:-90:0.5cm and 1cm);
		\draw[dashed, red] (0,1) arc (90:270:0.5cm and 1cm);
		
		\draw (0,0) circle (1cm);
		
		% Vẽ 2 cái dấu chấm màu đỏ
		\filldraw[red] (0,1) circle  (0.05); %add fill=, and draw= to have separate colours
		\filldraw[red] (0,-1) circle (0.05);
		
		\shade[ball color=blue!10!white,opacity=0.20] (0,0) circle (1cm);
		
	\end{tikzpicture}
	% Note: nhìn tưởng hình 3D nhưng thực chất là vẽ 2-D để tạo illusion!
\end{center}
\end{figure}

%\newpage
%\vspace{4in}

%%Various Line Thickness
\begin{figure}

\begin{tikzpicture}
	\draw[ultra thick] (0,3) -- (2,3);
	\draw[very thick] (0,2.5) -- (2,2.5);
	\draw[thick] (0,2) -- (2,2);
	\draw[thin] (0,1.5) -- (2,1.5);
	\draw[very thin] (0,1) -- (2,1);
	\draw[ultra thin] (0,.5) -- (2,.5);
	
	\draw node at (3, 3) {Ultra Thick};
	\draw node at (3, 2.5) {Very Thick};
	\draw node at (3, 2) {Thick};
	\draw node at (3, 1.5) {Thin};
	\draw node at (3, 1) {Very Thin };
	\draw node at (3, 0.5) {Ultra Thin};
\end{tikzpicture}

\caption{Line thickness}
\end{figure}

\begin{figure}

\begin{tikzpicture}[
	SIR/.style={rectangle, draw=red!60, fill=red!5, very thick, minimum size=5mm},
]
	
	%Nodes
	% (Susceptible) is the name of the node
	\node[SIR]    (Susceptible)                              {Susceptible $S(t)$};
	\node[SIR]    (Infectious)       [below=of Susceptible] {Infectious $I(t)$};
	\node[SIR]    (Recovered)       [below=of Infectious] {Recovered $R(t)$};
	
	%Lines
	\draw[->, very thick] (Susceptible.south)  to node[right] {$a$} (Infectious.north);
	\draw[->, very thick] (Infectious.south)  to node[right] {$b$} (Recovered.north);
	\draw[->, very thick] (Recovered.east) .. controls  +(right:7mm) and +(right:7mm)   .. (Susceptible.east);

\end{tikzpicture}
\end{figure}

\end{document}
