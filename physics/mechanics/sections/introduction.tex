\chapter{Introduction to Mechanics}

\section{What is Mechanics?}

\href{https://en.wikipedia.org/wiki/Mechanics}{Mechanics} (cơ học, from Ancient Greek 'of machines') is the area of physics concerned with the relationships between force, matter, and motion among physical objects. Forces applied to objects may result in displacements, which are changes of an object's position relative to its environment.

Mechanics can be divided into: Kinematics (động học) and Dynamics (động lực học).

\textbf{Kinematics} (chuyển động học, hay động học chất điểm) describes the motion of points (\textbf{chất điểm}), bodies (objects), and systems of bodies (groups of objects) \textbf{without} considering the forces that cause them to move.

In physics, \href{https://en.wikipedia.org/wiki/Dynamics_(mechanics)}{dynamics} (động lực học) is the study of forces and their effect on motion.

In the context of Kinematics, acceleration will always be constant.

\section{Linear motion}

Linear motion (or 1D motion, one dimensional motion) là chuyển động thẳng.

The linear motion can be of two types: \textbf{uniform linear motion}, with constant velocity (zero acceleration); and non-uniform linear motion, with variable velocity (non-zero acceleration).
