\chapter{Non-contact forces}

\section{Gravitational force}

Objects with mass exert \hl{gravitational forces} on each other. Gravitational forces are always attractive. Gravity is a \hl{non-contact} force that does not require objects be touching in order to act. Instead, the gravitational force can be modeled as acting through a \textbf{gravitational field}.

The strength of the gravitational forces two objects exert on each other depends on the \textbf{masses} of both objects, as well as the \textbf{distance} between their centers of mass.

\begin{itemize}
  \item \textbf{Fundamental forces}: basic interactions that cannot be explained by other interactions (gravity, electromagnetism, strong nuclear, and weak nuclear).
  \item \textbf{Non-fundamental forces}: friction, tension, and the normal force. These are complex interactions that arise from the fundamental forces, particularly electromagnetism, at a microscopic level.
\end{itemize}
