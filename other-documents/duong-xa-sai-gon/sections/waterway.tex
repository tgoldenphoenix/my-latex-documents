\chapter{Kênh rạch, sông, cầu}

\section{Kênh Tham Lương}

Sông Vàm Thuật – Bến Cát – Trường Đay – [Kênh Tham Lương](https://vi.wikipedia.org/wiki/S%C3%B4ng_V%C3%A0m_Thu%E1%BA%ADt_%E2%80%93_B%E1%BA%BFn_C%C3%A1t_%E2%80%93_Tr%C6%B0%E1%BB%9Dng_%C4%90ay_%E2%80%93_K%C3%AAnh_Tham_L%C6%B0%C6%A1ng_%E2%80%93_R%E1%BA%A1ch_N%C6%B0%E1%BB%9Bc_L%C3%AAn) – Rạch Nước Lên là một tuyến sông dài 30 km tại Thành phố Hồ Chí Minh.

Tuyến đường thủy này kết nối sông Sài Gòn với sông Bến Lức, chảy qua 8 quận huyện: Bình Thạnh, Gò Vấp, Quận 12, Tân Bình, Tân Phú, Bình Tân, Bình Chánh, Quận 8, tạo thành một đường vành đai bao bọc phía bắc và phía tây nội ô thành phố.

Bắt đầu từ sông Vàm Thuật là một nhánh của sông SG, ngay [Phù Châu Miếu](https://maps.app.goo.gl/ny5ymopuuYSy99Kv6), tới khoảng cầu Bình Điền.

Các cây cầu lớn bắc qua tuyến sông: cầu Ông Đung, cầu Võng, Cầu Dừa, cầu An Phú Đông, cầu An Lộc, cầu Bến Phân, cầu Trường Đai, cầu Chợ Cầu, cầu Tham Lương, cầu Bưng, cầu Tân Kỳ Tân Quý, cầu Bình Thuận, cầu Sông Chùa, cầu Bà Hom, cầu Đường A, cầu An Lập, cầu An Lạc, cầu Nước Lên.

\section{Kênh Tàu Hủ \& Rạch Bến Nghé \& Kênh Đôi \& Kênh Tẻ}

Kênh Tàu Hủ -> rạch Bến Nghé
Kênh Đôi -> Kênh Tẻ
=> **Tàu Hủ** phải đi **Đôi** với nước gừng. Nếu không thì sẽ **Tẻ** nhạt (chú ý order). Bến Nghé nằm dọc địa phận Q1 nơi có phường Bến Nghé (mặc dù p. Bến Nghé không thực sự nằm giáp với kênh.)

\section{Kênh Tàu Hủ}

Còn có thể gọi là **Kênh Chợ Lớn** vì nó nằm dọc Quận 5, 6. Mà Chợ Lớn nổi tiếng món tàu Hủ (mnemonic).

Kênh này dài khoảng 6 km, từ nơi giao với kênh Bến Nghé, kênh Tẻ, kênh Đôi chảy cho đến kênh Lò Gốm và rạch Ruột Ngựa ngay tại [Cầu Lò Gốm](https://maps.app.goo.gl/RSWp86s2Cracugat5). Kênh là ranh giới giữa Quận 5, Quận 6 (phía bắc kênh) và Quận 8 (phía nam kênh), có đại lộ Võ Văn Kiệt chạy dọc theo bờ bắc.

Kênh Lò Gốm sau đó chảy về và hợp với kênh Đôi.

Các cây cầu bắc qua kênh (theo thứ tự từ hướng trung tâm thành phố) bao gồm: cầu Chữ Y, cầu Nguyễn Tri Phương, cầu Chà Và.

\section{Rạch Bến Nghé}

Các cây cầu bắc qua rạch Bến Nghé: cầu Khánh Hội, cầu Mống, cầu Calmette, cầu Ông Lãnh và cầu Nguyễn Văn Cừ.

**Cầu Khánh Hội** kết nối đường Tôn Đức Thắng thuộc Quận 1 với đường Nguyễn Tất Thành thuộc Quận 4, nằm trước Bến Nhà Rồng.

\section{Kênh Đôi}

Các cây cầu bắc qua: cầu Chữ Y, cầu Chánh Hưng, cầu Nhị Thiên Đường.

Kênh Đôi là một con kênh đào dài 8,5 km, chảy hoàn toàn trong địa phận Quận 8, Thành phố Hồ Chí Minh.

Kênh Đôi nối từ ngã tư nơi giao với kênh Tàu Hủ, kênh Bến Nghé và kênh Tẻ đến ngã ba nơi giao với kênh Lò Gốm và sông Bến Lức. Kênh chạy song song với kênh Tàu Hũ, được chính quyền thực dân Pháp cho đào vào thập niên 1910, đến năm 1919 thì hoàn thành.

\section{Kênh Tân Hóa – Lò Gốm}

Kênh Tân Hóa – Lò Gốm, trước đây gọi là Rạch Tân Hóa – Lò Gốm, bắt đầu từ [gần đường Hòa Bình](https://maps.app.goo.gl/NtjEkd4EM8Df9p726) (Quận 11) đến ngã ba kênh Tàu Hủ – kênh Lò Gốm, đi qua 3 quận là quận Tân Phú, Quận 11 và Quận 6.

Kênh gồm 3 đoạn với 3 tên gọi khác nhau:

1. Đoạn từ đường Hòa Bình về đến ngã ba rạch Bến Trâu (cầu Ông Buông) được gọi là **rạch Tân Hóa** (có đường kênh Tân Hóa chạy dọc bờ kênh).
2. Đoạn tiếp theo đến đường Lê Quang Sung được gọi là **rạch Ông Buông**. Đoạn này rất ngắn.
3. Đoạn còn lại cắt ngang kênh Tàu Hủ tại cầu Lò Gốm rồi chảy đổ vào kênh Đôi được gọi là **rạch Lò Gốm** (có đường Lò Gốm chạy dọc bờ kênh).

\section{Sông Sài Gòn}

**Cầu Phú Mỹ** là cây cầu dây văng lớn nhất Thành phố Hồ Chí Minh bắc qua sông Sài Gòn nối Quận 7 và thành phố Thủ Đức, nằm trên tuyến đường vành đai 2 (Thành phố Hồ Chí Minh).

Cầu giúp kết nối Khu đô thị mới Thủ Thiêm và Khu đô thị Phú Mỹ Hưng, nối Quận 7 với thành phố Thủ Đức.

**Cầu Ba Son** là cây cầu dây văng bắc qua sông Sài Gòn, nối Quận 1 với thành phố Thủ Đức tại Thành phố Hồ Chí Minh, Việt Nam. Ban đầu cầu có tên là **cầu Thủ Thiêm 2**, tuy nhiên vào ngày 9 tháng 12 năm 2022, Hội đồng nhân dân Thành phố Hồ Chí Minh quyết định đặt tên mới là cầu Ba Son, theo tên của xưởng đóng tàu Ba Son xưa.

**Cầu Thủ Thiêm** là một cây cầu bắc qua sông Sài Gòn, nối liền quận Bình Thạnh và thành phố Thủ Đức, Thành phố Hồ Chí Minh. Cầu có 6 làn xe, nối Khu đô thị mới Thủ Thiêm và trung tâm hiện hữu của thành phố và được thông xe vào năm 2005.

Cầu nằm về phía tây-nam của Vinhomes Central Park. Tuy tên gọi là cầu Thủ Thiêm nhưng cầu Ba Son mới là cầu nằm gần hầm Thủ Thiêm nhất.

**Cầu Sài Gòn** (trước năm 1975 còn được gọi là **cầu Tân Cảng**) là một trong những cây cầu bắc qua sông Sài Gòn nối đường Điện Biên Phủ (quận Bình Thạnh) với đường Võ Nguyên Giáp (thành phố Thủ Đức), Thành phố Hồ Chí Minh. Cho đến khi đường hầm Thủ Thiêm được xây dựng xong thì đây vẫn là cửa ngõ chính để vào nội ô Thành phố Hồ Chí Minh từ các tỉnh miền Trung và miền Bắc Việt Nam.

Cầu nằm về phía Bắc của Vinhomes Central Park. Cầu Sài Gòn 2 song song với cầu Sài Gòn 1.

\section{Cầu Bình Triệu \& Bình Lợi}

**Cầu Bình Triệu** là tên của hai cây cầu bắc qua sông Sài Gòn trên tuyến đường quốc lộ 13, nối liền quận Bình Thạnh và thành phố Thủ Đức, Thành phố Hồ Chí Minh.

**Cầu Bình Lợi** là một cây cầu bắc qua sông Sài Gòn trên tuyến đường Phạm Văn Đồng tại Thành phố Hồ Chí Minh, nối liền quận Bình Thạnh và thành phố Thủ Đức.

**Cầu Bình Phước** là một trong những cây cầu bắc qua sông Sài Gòn. Cầu nằm trên Quốc lộ 1, giáp ranh giữa Quận 12 và thành phố Thủ Đức.

[Cầu Bình Điền](https://maps.app.goo.gl/pF72wAq83tzkYenu9) bắc qua sông chợ đệm về miền Tây.

\section{Chợ Bình Điền, Cầu Bình Điền}

[Cầu Bình Điền](https://maps.app.goo.gl/sy5A3wmYU3Y1tnzy5) là một cây cầu bắc qua sông Chợ Đệm tại huyện Bình Chánh.

Cầu nằm ở cửa ngõ phía tây nam Thành phố Hồ Chí Minh, trên tuyến giao thông quan trọng đi các tỉnh thành vùng Đồng bằng sông Cửu Long.

Gần đó là [Chợ Đầu mối Bình Điền](https://maps.app.goo.gl/sto3C6XJkDWJQekF7)

[Cầu bình triệu](https://maps.app.goo.gl/Mn1hzNvUgaoJQjM78) và [cầu bình lợi](https://maps.app.goo.gl/tpG5Zve4rCjU9XRk8) bắc qua sông Sài Gòn đi vào Thủ Đức

\section{Sài Gòn về Bến Tre Xe máy}

Qua [cầu bình điền](https://maps.app.goo.gl/FyskWTEQ4gJ5DTCS7) rẽ phải vào đường Nguyễn Hữu Trí
