\chapter{Tên đường}

\section{Một số lưu ý}

Khu Q1, Q3 có rất nhiều bảng cấm quay đầu ô tô cần chú ý.

\section{Ngã tư Từ Dũ}

Còn gọi là ngã ba Cao Thắng - Hồng Thập Tự; Nằm ngay bệnh viện Từ Dũ. Là nơi giao nhau của: NTMK, Cao Thắng, Cống Quỳnh.

Near: TPBank, Paris Baguette, Đ. Nguyễn Thiện Thuật

\section{Đ. Cao Thắng}

Đường đi từ con hẻm ngay Hồ Kỳ Hoà (Đông Hồ Eden) chạy ra 3/2, Điện Biên Phủ, Nguyễn Đình Chiểu \& kết thúc ở ngã tư Từ Dũ.

On the road: Cà Phê Sân Vườn Đông Hồ, Mega GS Cinemas Cao Thắng, Paris Baguette Cao Thắng, chợ Bàn Cờ, Xuân Vũ audio

\section{Đ. Lê Lợi, Nguyễn Huệ}

\hl{Đường Lê Lợi} hay Đại lộ Lê Lợi là một tuyến đường tại trung tâm Quận 1, nối từ chợ Bến Thành đến Nhà hát thành phố.

Tuyến đường này bắt đầu từ công trường Quách Thị Trang trước chợ Bến Thành, giao cắt với các tuyến đường Phan Bội Châu, Nguyễn Trung Trực, Nam Kỳ Khởi Nghĩa, Pasteur, Nguyễn Huệ và kết thúc tại đường Đồng Khởi trước Nhà hát Thành phố Hồ Chí Minh.

Nằm trên đường: Công trường Lam Sơn

\threestars

\hl{Đường Nguyễn Huệ} là một tuyến đường tại Quận 1, trung tâm Thành phố Hồ Chí Minh, chạy từ Trụ sở Ủy ban nhân dân Thành phố đến Bến Bạch Đằng, bờ sông Sài Gòn. Hiện nay chính giữa con đường này là một quảng trường đi bộ rộng 27 m được đưa vào sử dụng từ năm 2015, là quảng trường đi bộ đầu tiên của Việt Nam.

Đường Nguyễn Huệ bắt đầu từ đường Lê Thánh Tôn ngay đối diện Trụ sở Ủy ban nhân dân Thành phố Hồ Chí Minh, cắt qua các con đường Lê Lợi, Nguyễn Thiệp, Tôn Thất Thiệp, Mạc Thị Bưởi, Huỳnh Thúc Kháng, Ngô Đức Kế, Hải Triều và kết thúc tại đường Tôn Đức Thắng đối diện với công viên Bến Bạch Đằng.

\section{Đ. Hàm Nghi}

Đường Hàm Nghi hay Đại lộ Hàm Nghi là một tuyến đường tại trung tâm Quận 1, nối từ chợ Bến Thành đến Bến Bạch Đằng.

Tuyến đường này bắt đầu từ công trường Quách Thị Trang trước chợ Bến Thành, giao cắt với các tuyến đường Phó Đức Chính, Nam Kỳ Khởi Nghĩa, Pasteur, Tôn Thất Đạm, Hồ Tùng Mậu và kết thúc tại đường Tôn Đức Thắng, đối diện với Bến Bạch Đằng, bờ sông Sài Gòn.

Nằm trên đường: Bitexco Financial Tower, Bánh Mì Như Lan, Trường Đại học Ngân hàng TP/.HCM - Trụ sở chính

\section{Đ. Trần Xuân Soạn}

Đường đi từ cầu Rạch Ông dọc theo Kênh Tẻ và kết thúc dưới chân cầu Tân Thuận.

\section{Khu xóm ga Sài Gòn}

\hl{Đường Trần Văn Đang}, đoạn đường ngắn nhưng "chuyên chở" nhiều cảnh đời dân nhập cư tìm giấc mơ ở thành phố. Chạy dọc Ga tàu lửa Sài Gòn. Hẻm 79 Trần Văn Đang.

Lý giải về cái tên \hl{Cống Bà Xếp}, bà Nguyễn Thị Lệ (68 tuổi), một người dân sinh sống trong khu này cho biết: “Thời ông bà, cha mẹ tui kể lại là xưa có một bà không rõ tên gì, bà là vợ của một ông sếp làm việc ở ga Hòa Hưng thời Tây nên mọi người hay gọi là Bà Xếp”.

Theo ký ức của bà Lệ từ lời kể truyền miệng của ông bà, Bà Xếp thấy dân quanh đây sống cảnh “mưa nhỏ chút thôi cũng bị ngập liên miên”, không biết bao nhiêu đứa con nít bị nước cuốn mất xác mỗi bận mưa, nước dâng, nên đã bỏ tiền túi ra làm cái cống thoát nước cho bà con được sinh hoạt thoải mái hơn.

“Tên cống cũng là đặt theo tên người dân hay gọi bà Xếp, “đáng lý ra là “Sếp”, nhưng thời đó đâu có nhiêu người biết chữ, người ta cứ viết là “Xếp” nên thành ra như vậy luôn”, bà Lệ lập luận.

Khu vực Cống Bà Xếp bao gồm những con đường như Trần Văn Đang, Kỳ Đồng, Rạch Bùng Binh hay Nguyễn Thông… ngày trước vốn chỉ là những con hẻm nhỏ, hẹp và ngoằn nghèo

Khu này xưa là nơi giang hồ, dân nghèo tụ tập.

\section{Đại thế giới}

Đại Thế Giới là sòng bạc lớn nhất Đông Dương trước năm 75.

Đại Thế giới nằm ở vị trí hiện nay là Trung tâm Văn hóa Quận 5 (địa chỉ số 105 đường Trần Hưng Đạo) và Công viên nước Đại Thế giới (địa chỉ số 1106 đường Võ Văn Kiệt) tại phường 6, Quận 5, Thành phố Hồ Chí Minh. Đây cũng là nơi tổ chức ngày hội Tết Nguyên tiêu hàng năm của Thành phố Hồ Chí Minh.

\section{Đ. Lý Tự Trọng}
%anki

Đường Lý Tự Trọng (đường Gia Long) là tuyến đường \textbf{một chiều} tại Quận 1, đi từ ngã sáu Phù Đổng đến đường Tôn Đức Thắng (trước nhà nguyện cổ).

Đường từng chứng kiến cuộc dậy sóng chính biến vào ngày 1-11-1963.

Nằm trên đường: Bảo tàng Thành phố Hồ Chí Minh (trước là \textbf{Dinh Gia Long}), Tòa án Nhân dân Tp. Hồ Chí Minh, Thư viện Khoa học Tổng hợp TP/.HCM, Bệnh viện Nhi Đồng 2

\section{Đ. Sương Nguyệt Anh}

Con đường Sương Nguyệt Anh ấy ngắn thôi, chạy song song với đường Nguyễn Thị Minh Khai (xưa là Hồng Thập Tự) và đường Bùi Thị Xuân, đồng thời nối liền đường Tôn Thất Tùng (Bùi Chu) với đường Cách Mạng Tháng Tám (Lê Văn Duyệt). Đây là đường một chiều chạy từ Tôn Thất Tùng qua CMT8.

 Con đường này có từ năm 1926, hợp cùng đường Duranton (Bùi Thị Xuân) tạo thành một ô chữ nhật chỉ gồm các villa của các "danh gia - vọng tộc" thời Pháp.

Trong đó, nhà số 102 là biệt thự và cũng là phòng mạch của bác sĩ Phạm Ngọc Thạch. Sau năm 1955, con đường đổi tên thành Sương Nguyệt Anh - con gái của cụ Đồ Chiểu và là một nhà thơ, nhà báo nổi tiếng Sài Gòn đầu thế kỷ 20.

 Nằm trên đường: 2 hàng cổ thụ, biệt thự pháp.

\section{Đ. Nguyễn Du}

**Vị trí**: Đường Nguyễn Du ở quận 1, bắt đầu từ đường Nguyễn Bỉnh Khiêm gần Thảo cầm viên đến đường CMT8 gần ngã sáu Sài Gòn, dài khoảng 2000m, qua ngã tư Tôn Đức Tnắng, các ngã ba Chu Mạnh Trinh bên trái, Mạc Đĩnh Chi bên phải, các ngã tư Hai Bà Trưng, Đồng Khởi, Nguyễn Thị Minh Khai, Nam Kỳ Khởi Nghĩa, ngã ba Nguyễn Trung Trực bên trái, Công Chúa Huyền Trân bên phải, ngã tư Trương Định, ngã ba Đặng Trần Côn bên trái.

Đường này lưu thông một chiều theo hướng từ Hai Bà Trưng đến CMT8 và hai chiều đoạn từ Tôn Đức Thắng đến đường Hai Bà Trưng, nhưng cấm xe ô tô tải nặng.

Lịch sử: Đường này thời Pháp thuộc là hai đường nối nhau. Đường thứ nhất có tên Lucien Mossard từ đường Nguyễn Bỉnh Khiêm đến đường Tự Do (Đồng Khởi); đường thứ hai là đường Taberd. Từ năm 1955 chính quyền Sài Gòn nhập hai đường làm một và đổi tên là đường Nguyễn Du cho đến nay

Nằm trên đường: Nhạc viện Thành phố Hồ Chí Minh, CV Tao Đàn, Tổng lãnh sự quán Hàn Quốc, Tòa án Nhân dân, Viện Kiểm sát Nhân dân, công trường Công Xã Paris, Bệnh viện Nhi Đồng 2, Trung tâm Mục vụ Tổng Giáo phận Sài Gòn.

\section{Đ. Phạm Ngọc Thạch}

Trước 1985 đây là đường **Duy Tân**. Duy Tân là vua của nhà Nguyễn.

"Trả lại em yêu khung trời đại học, con đường Duy Tân cây dài bóng mát"

Nằm trên đường: hồ Con Rùa, Nhà văn hóa Thanh niên, Thành Đoàn Thành phố, quán cà phê, trường đại học Kinh Tế CS 1, ĐH Kiến Trúc, và những ngôi nhà "hồn muôn năm cũ", nhà Trịnh Công Sơn (47C Duy Tân)

**Diamond Plaza** là một tòa cao ốc ở Thành phố Hồ Chí Minh. Tòa cao ốc này gồm 22 tầng, chia làm 2 bên: Khu thương mại và Khu căn hộ. Tòa nhà được xây xong vào năm 1999, tọa lạc ở góc đường Lê Duẩn và Phạm Ngọc Thạch, nằm phía sau lưng Nhà thờ Đức Bà Sài Gòn, ngay trung tâm thành phố.

\section{Đ. Trần Nhật Duật}

Các con đường được đặt tên theo quan lại và tướng dưới triều nhà Trần nằm ở Phường Tân Định, Q1:
Trần Nhật Duật\
Trần Khánh Dư

\section{Đ. Nguyễn Đình Chiểu}

d

\section{Xóm Vườn Chuối}

Ngay tại trung tâm quận 3, nằm bên cạnh đường Nguyễn Đình Chiểu và Nguyễn Thượng Hiền là chợ Vườn Chuối, thuộc xóm Vườn Chuối ngày xưa. Chạy dọc bên cạnh chợ là một con đường cùng tên, dài khoảng hơn 400 mét.

Đường Vườn Chuối nhỏ nhưng đi được hai chiều, thuộc khu vực Cư xá Đô Thành. Từ xưa đến nay, con đường này có nhiều nhà may, tiệm quần áo cũ, áo cưới, quán ăn,… nằm san sát nhau.

Những người sống trong khu vực này cho biết, xóm Vườn Chuối nằm trong khu tập trận. Người dân không ai dám khai khẩn làm gì, chỉ trồng chuối thành vườn rồi từ đó truyền tai nhau gọi tên là xóm Vườn Chuối. Năm 1955, chính quyền Sài Gòn dùng chính tên mà người dân hay gọi để đặt cho chợ và con đường Vườn Chuối ngày nay.

Gần đó: Trường THCS Bàn Cờ

\section{Đ. Trần Bình Trọng}

đường thuộc Quận 5 kéo dài từ đường Võ Văn Kiệt đến hẻm 80 đường Hồ Thị Kỷ.

Trần Bình Trọng là con trai của tướng Lê Trần và công chúa Chiêu Thánh. Ông là người có công lớn hộ giá bảo vệ cho hai vua Trần (Trần Thánh Tông và Trần Nhân Tông) trong cuộc kháng chiến chống quân Nguyên-Mông lần thứ hai. Ông hy sinh khi chặn quân Nguyên ở bãi Thiên Mạc, được truy phong làm Bảo Nghĩa Vương.

Trước khi bị giết hại ông đã để lại cho đời sau câu nói nổi tiếng : “Ta thà làm ma nước Nam chứ không thèm làm vương đất Bắc”.

\section{Chợ Bàn Cờ}

Không xa chợ Vườn Chuối còn có chợ Bàn Cờ (Q3), cũng có đường Bàn Cờ, tiếng tăm thì chợ Vườn Chuối được nhiều người biết hơn. Tuy vậy về tầm vóc thì chợ Bàn Cờ nhộn nhịp hơn và xứng đáng là trung tâm ẩm thực đường phố của quận 3 nói riêng và thành phố nói chung. Với phố ẩm thực ban đêm ở đường Nguyễn Thiện Thuật có từ trước 1975, cùng với nhiều món ăn được vinh danh trên các báo trong và ngoài nước: như Bánh mì Hòa Mã, bánh mì thịt nướng, bánh Huế, cháo Tiều, bánh tráng nướng, hủ tíu bò viên, bánh cuốn…

Chợ còn nổi tiếng bán quần áo thời trang, đồ si đa.

\section{Đ. Trần Quốc Thảo}

Vị trí: Đường nằm trên địa bàn quận 3, bắt đầu từ đường Võ Văn Tần kết thúc ở cầu Lê Văn Sĩ.

Bệnh Viện Tai Mũi Họng Tp. HCM, UBND Quận 3, Tổng Lãnh Sự Quán Thái Lan \& Ấn Độ, Tòa Tổng Giám Mục Sài Gòn

\section{Đ. Lạc Long Quân -- Phú Thọ}

Từ nút giao [Lý Thường Kiệt] gần ngã tư Bảy Hiền, khi chạy tới chỗ giao với đường Xóm Đất thì đổi tên thành đường Phú Thọ. Đường Phú Thọ tiếp tục chạy ra cầu vượt Cây Gõ.

Vòng xoay Đầm Sen là nơi giao với đường Hòa Bình--Ông Ích Khiêm.

Đường Âu Cơ: Từ vòng xoay Lê Đại Hành chạy ra đổ vào đường Trường Chinh tại mũi tàu Trường Chinh.

\section{Đ. Lũy Bán Bích -- Tân Hóa}

Đường Tân Hóa bắt đầu dưới chân cầu Ông Buông rẽ nhánh từ Hồng Bàng khi đi đến kênh Tân Hóa thì đổi tên thành Lũy Bán Bích chạy đến khi đổ vào đường Âu Cơ tại [mũi tàu Âu Cơ.

\section{Chợ Bàu Sen, Chợ Bàu Cát}

Chợ Bàu Sen nằm ở đường Nguyễn Trãi, Chợ Lớn. Kế đó là trường tiểu học Bàu Sen. Trước đây là một cái ao sen.

Chợ Bàu Cát nằm trên đường Bàu Cát, quận Tân Bình.

\section{Trần Quang Diệu, Trần Huy Liệu \& Trần Khắc Chân}

Trần Huy Liệu là đường 1 chiều Bắc-Nam bắt đầu từ một nhánh rẽ của đường Hoàng Văn Thụ ngay nhà thờ [Gx Phú Nhuận](https://maps.app.goo.gl/kmQvEQQwiT3J4kpK9). Khi tới địa phận Q3 thì đổi tên thành Trần Quang Diệu (ngay [Bv An Sinh](https://maps.app.goo.gl/xYC3gX1RD21gyCPe7)). Số nhà được đánh số giảm dần theo chiều xe chạy. Số 1 bắt đầu từ đường Trường Xa; reset số 1 tại Bv An Sinh.

Trần Huy Liệu là một nhà báo cách mạng. Trần Quang Diệu là tướng nhà Tây Sơn. Trần Khát Chân là tướng nhà Trần.

\section{Hoàng Văn Thụ}

kéo dài từ Ngã Tư Phú Nhuận đến Ngã Tư Bảy Hiền.

\section{Đ. Bùi Hữu Nghĩa}

Là một đường thuộc Quận 5 bắt đầu từ đường Đào Tấn (kế bên kênh Tàu Hủ) đến đường Nguyễn Trãi (kế bên Bv Nguyễn Trãi).

Bùi Hữu Nghĩa hay Thủ khoa Nghĩa, trước có tên là là Bùi Quang Nghĩa, hiệu Nghi Chi; là quan nhà Nguyễn, là nhà thơ và là nhà soạn tuồng Việt Nam.

Chợ Hòa Bình, chùa vạn phật

\section{Ngã tư Bình Thái}

Là nơi giao nhau giữa xa lộ Hà Nội và đường Tây Hòa, đường số 1 vào cảng Trường Thọ. Nằm ngay Trạm Biến Áp 35kv Thủ Đức

\section{Ngã tư Thủ Đức \& Ngã ba Trạm 2}

Cầu vượt Ngã tư Thủ Đức là một trong những nút giao thông quan trọng của TP Thủ Đức, giao cắt giữa các tuyến đường Xa lộ Hà Nội, Võ Văn Ngân, Lê Văn Việt. Nằm gần Trường Đại học Sư phạm Kỹ thuật Thành phố Hồ Chí Minh.

Còn Ngã ba Thủ Đức (hay [ngã ba Trạm 2](https://maps.app.goo.gl/DLAhsvb2yxEwnGBM6)) nằm cách đó 4km về hướng Đông Bắc ngay Suối Tiên. Là nơi giao nhau giữa xa lộ Hà Nội và xa lộ Đại Hàn. Từ Ngã ba Trạm 2 về đầu bắc là thuộc QL1. Trước đó là QL52.

\section{Ngã ba Tân Vạn}

Là nút giao thông chạy qua địa bàn phường Bình Thắng, Thị xã. Dĩ An, tỉnh Bình Dương và là một cửa ngõ quan trọng ở phía Đông Bắc, nối Tp/.HCM, Đồng Nai và các tỉnh lân cận khu vực này. Tuyến đường bộ vô cùng quan trọng trong việc lưu thông hàng hóa và di chuyển của người dân ở đây.

Là nơi giao nhau giữa: xa lộ Hà Nội, DT743A. Nằm gần: Nhà Thờ Giáo Xứ Nghĩa Sơn

\section{Ngã ba Vũng Tàu}

Dù nay đã là ngã tư nhưng tên gọi **Ngã ba Vũng Tàu** vẫn được nhiều người nhớ đến. Là nơi giao nhau giữa: QL1A, QL51, DT11. QL51 là hướng đi Vũng Tàu nên có tên gọi như vậy. Nằm gần: BigC Đồng Nai

\section{Ngã ba Tam Hiệp}

Là nơi giao nhau giữa xa lộ Hà Nội và QL15. Nằm ngay Bùng Binh Tam Hiệp.

\section{Cầu Đồng Nai}

Là một cây cầu đường bộ quan trọng nằm tại km 1872 + 579 thuộc Quốc lộ 1, bắc qua sông Đồng Nai, nối thành phố Biên Hòa (tỉnh Đồng Nai) và thành phố Dĩ An (tỉnh Bình Dương). Cầu nằm gần Big C Đồng Nai.

Được xây dựng từ năm 1964, cây cầu hiện nay đã có dấu hiệu xuống cấp mặc dù là tuyến giao thông huyết mạch với hơn 44.000 lượt xe mỗi ngày. Có cảnh báo cầu có thể bị sập bất cứ lúc nào. Hiện tại, dự án xây dựng **cầu Đồng Nai 2** nằm song song với cây cầu đã được tiến hành để có thể thay thế một phần tải trọng cho cầu Đồng Nai cũ.

Còn các cây cầu khác như: Cầu Rạch Chiếc

\section{Xa lộ Hà Nội \& các nút giao}

Xa lộ Hà Nội] (hay còn gọi là Quốc lộ 52, trước đây Xa lộ Biên Hòa, đường **Võ Nguyên Giáp**) nối liền Thành phố Hồ Chí Minh và Biên Hòa, Đồng Nai được xây dựng từ năm 1957 đến năm 1961, do các chuyên gia Hoa Kỳ đầu tư và giúp đỡ xây dựng.

Con đường này dài 31 km, bắt đầu từ [Cầu Sài Gòn](https://maps.app.goo.gl/9uu565bB5yShE5FVA) quận Thủ Đức (nối tiếp đường Điện Biên Phủ); kết thúc là nút giao cắt với đường Nguyễn Ái Quốc tại [ngã 3 Chợ Sặt](https://maps.app.goo.gl/YUCwUidSuGHET5Ag9), phường Tân Biên, thành phố Biên Hòa. Đây là một trong những con đường cửa ngõ dẫn để vào nội ô Thành phố Hồ Chí Minh khi đi từ các tỉnh miền Đông Nam Bộ, Trung Bộ và Bắc Bộ.

Trên xa lộ này có hai cây cầu lớn bắc ngang là cầu Sài Gòn (dài 982 m) bắc qua sông Sài Gòn và cầu Đồng Nai (dài 453 m) bắc qua sông Đồng Nai.

Một tên khác của con đường này là quốc lộ 52, thường được dùng để chỉ đoạn từ chân cầu Sài Gòn đến chỗ giao nhau với Quốc lộ 1 tại ngã ba Trạm 2. Hiện nay trên xa lộ Hà Nội có một đoạn Quốc lộ 1 đi qua, bắt đầu từ ngã ba Thủ Đức đến **ngã ba Hố Nai** hay còn gọi là ngã ba Chợ Sặt vì chợ Sặt thành phố Biên Hòa nằm gần ngã ba này (là **ngã ba Công viên 30 tháng 4**, giao với quốc lộ 1K, vượt quá ngã tư Tam Hiệp).

Ngày 12 tháng 7 năm 2023, Hội đồng nhân dân Thành phố Hồ Chí Minh thông qua nghị quyết về việc đổi tên đoạn Xa lộ Hà Nội từ cầu Sài Gòn đến ngã tư Thủ Đức với chiều dài 7,79 km thành đường Võ Nguyên Giáp.

\section{Xa lộ Đại Hàn}

Xa lộ Đại Hàn là tên gọi cho đoạn Quốc lộ 1 từ [ngã ba Trạm 2](https://maps.app.goo.gl/yRFEqfMu5FhYhwQC7) đến ngã ba An Lạc, quận Bình Tân, đi qua địa phận Thành phố Hồ Chí Minh và tỉnh Bình Dương, dài 43,1 km.

[Ngã ba An Lạc](https://maps.app.goo.gl/sL3r8MRSrFNq56fi6) là nơi giao nhau giữa QL1A và đại lộ Đông Tây.\
[Ngã tư An Lạc](https://maps.app.goo.gl/1ryiLuZNQmJ9cPG17) nằm gần đó Kinh Dương Vương x Hồ Học Lãm

Xa lộ này được công binh quân đội Đại Hàn Dân Quốc xây dựng năm 1969 - 1970 sau sự kiện Tết Mậu Thân trong Chiến tranh Việt Nam với tư cách là đồng minh của Việt Nam Cộng Hoà nhằm làm đường vành đai bảo vệ sân bay Tân Sơn Nhứt và Sài Gòn và ngăn cách giữa Sài Gòn với các lực lượng Việt Cộng ở Củ Chi, Hóc Môn.

Có 2 đoạn:

- Đoạn Trạm 2 - An Sương
- Đoạn An Sương - An Lạc (lộ giới 120m) m

Hiện nay, đoạn từ ngã ba Trạm 2 đến ngã tư An Sương của Xa lộ Đại Hàn dài 21,1 km nằm trong đường Xuyên Á nối Thành phố Hồ Chí Minh và Phnom Penh, Bangkok. Phần tiếp theo của đường xuyên Á theo quốc lộ 22 qua Củ Chi đến cửa khẩu Mộc Bài tiếp giáp với Cambodge Tổng chiều dài của đường xuyên Á đoạn trong Thành phố Hồ Chí Minh (đến Củ Chi) dài 43,6 km.

Các khu công nghiệp tập trung hiện nay của Thành phố Hồ Chí Minh chủ yếu nằm dọc theo hành lang xa lộ Đại Hàn (đường Vành đai 2), gồm các khu công nghiệp Tân Tạo, Tân Bình, Vĩnh Lộc, Tân Thới Hiệp, Linh Trung, Bình Chiểu, Thủ Đức và khu công nghệ cao, công viên phần mềm Quang Trung.

\section{Đại lộ Đông - Tây}

Đại lộ Đông – Tây hay **Đại lộ Võ Văn Kiệt – Mai Chí Thọ**, là một tuyến đường đi qua trung tâm Thành phố Hồ Chí Minh, là một trục đường mới ra vào phía Nam theo hướng Đông – Tây, nhằm giảm ách tắc giao thông cho cầu Sài Gòn và các trục chính trong thành phố. Tuyến đường này đáp ứng yêu cầu lưu thông cho các cảng của thành phố đi các nơi theo hướng Đông Bắc – Tây Nam và các tỉnh Đồng bằng sông Cửu Long, tạo trục giao thông sang Thủ Thiêm, và cải thiện môi trường ven kênh mà nó đi qua, tăng vẻ mỹ quan cho thành phố.

Con đường có tổng chiều dài gần 22 km, bắt đầu từ **ngã ba Cát Lái** (giao với xa lộ Hà Nội tại TP Thủ Đức) đến quốc lộ 1A (huyện Bình Chánh, ngay nhà tang lễ TP), đi qua TP Thủ Đức, quận 1, 4, 5, 6, 8, Bình Tân và huyện Bình Chánh.

Vị trí

Đại lộ chạy dọc theo kênh từ Quốc lộ 1 huyện Bình Chánh (ngay nhà tang lễ thành phố) đến ngã ba đường Yersin – Chương Dương gần cầu Calmette, quận 1; vượt sông Sài Gòn bằng hầm Thủ Thiêm; nối vào Đại lộ Mai Chí Thọ rồi sau đó giao với xa lộ Hà Nội tại Ngã ba Cát Lái, Thủ Đức.

Chiều dài toàn tuyến là 21,9 km, đi qua địa bàn các quận 1, 5, 6, 8, Bình Tân, huyện Bình Chánh và thành phố Thủ Đức, tạo thành một tuyến trục giao thông Đông – Tây, và kết nối hai đầu Đông Bắc – Tây Nam thành phố. Đại lộ Đông – Tây tạo điều kiện thuận lợi cho các phương tiện giao thông ra vào cảng Sài Gòn và từ đây đi các tỉnh miền Đông và miền Tây không phải đi vào trung tâm thành phố. Đây là con đường huyết mạch liên kết chặt chẽ các địa phương trong vùng kinh tế trọng điểm phía Nam.

\section{Đại lộ Mai Chí Thọ}

Đại lộ Mai Chí Thọ là một tuyến đường trục của Thành phố Hồ Chí Minh, kết nối cửa ngõ phía đông với trung tâm thành phố. Đường này là một phần của Đại lộ Đông – Tây dài 22 km từ ngã ba Cát Lái (thành phố Thủ Đức) đến Quốc lộ 1 (huyện Bình Chánh).

Đại lộ Mai Chí Thọ (từ Xa lộ Hà Nội về đến hầm vượt sông Sài Gòn, phía quận 2) dài 9 km, mặt đường rộng 140 m. Ngoài việc rút ngắn chiều dài từ Đông sang Tây, đại lộ này còn có những đường nối dẫn vào cảng Cát Lái, vành đai 2... giúp cho việc lưu thông qua đây dễ dàng hơn. Bên cạnh đó, tuyến đường đã thu hút hàng chục dự án bất động sản đổ về, khiến diện mạo đô thị, cư dân khu vực này thay đổi nhanh chóng.

\section{Đ. Tôn Đức Thắng}

Đường này bắt đầu từ vàm rạch Bến Nghé (đầu cầu Khánh Hội ngày nay) giao với đường Võ Văn Kiệt (Bến Chương Dương cũ), đi dọc bờ tây sông Sài Gòn cắt qua các con đường Hàm Nghi, Nguyễn Huệ, Đồng Khởi, Công trường Mê Linh đến xưởng đóng tàu Ba Son cũ (nay là vị trí chân cầu Ba Son). Tại đây đường có một khúc cua sang trái rồi tiếp tục đi thẳng cắt qua các con đường Nguyễn Siêu, Nguyễn Hữu Cảnh – Lê Thánh Tôn, Lý Tự Trọng, Nguyễn Trung Ngạn, Nguyễn Du và kết thúc tại ngã tư giao với đường Lê Duẩn, đối diện với đường Đinh Tiên Hoàng.

Đường Tôn Đức Thắng trước đây được nhiều người dân biết đến với 4 hàng cây xà cừ cổ thụ rợp bóng mát được người Pháp trồng cách đây hơn một thế kỷ. Tuy nhiên, vào năm 2017, để xây dựng cầu Thủ Thiêm 2 (nay là cầu Ba Son), thành phố đã cho đốn hạ, di dời hàng cây này.

\section{Công trường Mê Linh}

Công trường Mê Linh là một vòng xoay giao thông nằm ở Quận 1, Thành phố Hồ Chí Minh, Việt Nam, kế cận công viên Bến Bạch Đằng và sông Sài Gòn. Đây là giao điểm của sáu con đường, ở giữa có một hồ nước nhân tạo đặt tượng Trần Hưng Đạo từ trước năm 1975 (từ năm 1967 đến nay).

Trước Tết Nguyên Đán 2019, phía trước tượng còn có một lư hương để người dân thờ cúng nhưng nhà chức trách đã dời về Đền thờ Đức thánh Trần Hưng Đạo ở đường Võ Thị Sáu, Quận 1.

Mê Linh là kinh đô của Hai Bà Trưng. Ngay đó có đường Hai Bà Trưng đi tới cầu Kiệu.

Ngay đây còn có [tòa nhà Vietcombank Tower].

\section{Đ. Trần Hưng Đạo}

**Đường Trần Hưng Đạo** nối từ công trường Quách Thị Trang (Quận 1) đến đường Học Lạc (Quận 5, trước nhà thờ St Francis Xavier Church hay **nhà thờ cha Tam**) Đây là đại lộ nối liền hai thành phố Sài Gòn và Chợ Lớn cách đây một thế kỷ và ngày nay tuyến đường vẫn là một trục giao thông quan trọng của Thành phố Hồ Chí Minh.

**Ngã tư Đèn Năm Ngọn** Trần Hưng Đạo x Phùng Hưng.

Ngã 4 trần Hưng Đạo x Nguyễn Biểu có Cửa hàng Xăng dầu Petrolimex Số 18 ngày nào đi học cũng đi qua [here](https://maps.app.goo.gl/Pjx2wy8qorra76yA6).

Chợ đại quang minh, Phòng Cảnh sát Phòng cháy Chữa cháy và Cứu nạn Cứu hộ TP. Hồ Chí Minh

\section{Công trường Quách Thị Trang-Trần Nguyên Hãn}

Công trường Quách Thị Trang từng là một vòng xoay giao thông (bùng binh) ở Quận 1, Thành phố Hồ Chí Minh, Việt Nam, ngay mặt tiền chợ Bến Thành nổi tiếng. Chính giữa vòng xoay là nơi từng đặt hai bức tượng Quách Thị Trang và Trần Nguyên Hãn.

\section{Đường Phạm Văn Đồng}

Đường Phạm Văn Đồng], trước đây gọi là đường **Tân Sơn Nhất – Bình Lợi – Vành đai ngoài**, là một con đường thuộc tuyến đường vành đai 1 tại Thành phố Hồ Chí Minh, bắt đầu từ [Ngã năm Nguyễn Thái Sơn](https://maps.app.goo.gl/bRXUrCTFn8dxqNBg6) quận Gò Vấp đến [Ngã tư Linh Xuân](https://maps.app.goo.gl/4T4ybUf73kqYvoLN7) thuộc thành phố Thủ Đức, có tổng chiều dài 12,211 km. Đại lộ được đặt tên theo cố thủ tướng Việt Nam Phạm Văn Đồng.

Đường Phạm Văn Đồng dài 13,6 km, rộng 30-60 m, có tổng vốn đầu tư 340 triệu USD, được hoàn thành và đưa vào sử dụng vào cuối năm 2014. Đây là trục đường hướng tâm quan trọng của thành phố kết nối từ sân bay Tân Sơn Nhất đến vành đai 2, quốc lộ 1A, 1K qua các quận Tân Bình, Gò Vấp, Bình Thạnh, TP Thủ Đức và kết nối Bình Dương, Đồng Nai

\section{Đại lộ Nguyễn Văn Linh}

Đại lộ Nguyễn Văn Linh nhìn từ cầu Ông Lớn (ngay đại học RMIT)

Đường Nguyễn Văn Linh, tên gọi trước đây là đường Bắc Nhà Bè – Nam Bình Chánh, là một tuyến đường trục kết nối Quận 7 và huyện Bình Chánh ở phía nam Thành phố Hồ Chí Minh. Một phần của con đường này thuộc đường vành đai 2 Thành phố Hồ Chí Minh.

Đại lộ có tổng chiều dài 17,8 km, nối từ đường Huỳnh Tấn Phát, Quận 7 đến Quốc lộ 1 (đoạn đi qua Bình Chánh), kết nối với đường dẫn cao tốc Thành phố HCM - Trung Lương. Ở phí đầu giao với đường Huỳnh Tấn Phát là công viên Tân Thuận.
Đại lộ được quy hoạch lộ giới 60-120 m, gồm 10 làn xe, 10 cây cầu, với tổng kinh phí đầu tư khoảng 100 triệu USD. Đây là công trình hạ tầng lớn nhất và có ý nghĩa quan trọng nhất mà Công ty Liên doanh Phú Mỹ Hưng thực hiện.
Đây là công trình hạ tầng đô thị tầm vóc của Việt Nam và nó đã được đặt tên cố Tổng bí thư Đảng Cộng sản Việt Nam Nguyễn Văn Linh vào năm 2000.
Đại lộ Nguyễn Văn Linh được xem là tuyến đường huyết mạch có ý nghĩa đối với sự phát triển kinh tế cho khu vực phía Nam thành phố Hồ Chí Minh, kết nối với những công trình trọng điểm như: Khu chế xuất Tân Thuận, Khu đô thị Phú Mỹ Hưng, Nhà máy điện Hiệp Phước, Khu công nghiệp Hiệp Phước, Khu đô thị cảng Hiệp Phước, Cảng Hiệp Phước, Cầu Phú Mỹ và Khu đô thị mới Thủ Thiêm.

Được khởi công từ tháng 12/1996, vốn đầu tư 100 triệu USD, con đường được mệnh danh huyền thoại thời mở cửa, với trục giao thông đô thị huyết mạch dài và bề thế nhất TP HCM này chính thức đưa vào sử dụng từ 2007 sau 11 năm xây dựng. Đại lộ là xương sống của toàn bộ khu đô thị phía Nam thành phố kết nối với những công trình trọng điểm như khu chế xuất Tân Thuận, đô thị Phú Mỹ Hưng, nhà máy điện Hiệp Phước.

Các địa danh trên đường gồm:

1. Đại học RMIT
2. Khu đô thị Phú Mỹ Hưng siêu giàu nhà Cường Dollar.
3. Hồ Bán Nguyệt

\section{Kênh Nhiêu Lộc - Thị Nghè}

Kênh dài gần 9km, bắt đầu từ cửa cống hộp tại điểm giao giữa hai con đường Lê Bình và Út Tịch thuộc quận Tân Bình, chảy qua các các quận: Tân Bình, Quận 3, Phú Nhuận, Quận 1, Bình Thạnh và đổ ra sông Sài Gòn tại vị trí xưởng đóng tàu Ba Son cũ.

\section{Đ. Hoàng Sa - Trường Sa}

Hiện nay, hai con đường chạy dọc kênh Nhiêu Lộc – Thị Nghè được đặt theo tên hai quần đảo trên Biển Đông mà Việt Nam tuyên bố chủ quyền: đường Trường Sa bên tả ngạn (phía sân bay) dài 8,3 km và đường Hoàng Sa bên hữu ngạn (phía Q1) dài 7,4 km. Sau đó, trong giai đoạn 2013–2015, thành phố cũng cho xây mới 3 cầu bắc qua tuyến kênh là cầu Lê Văn Sỹ, cầu Kiệu và cầu Bông, cũng như xây dựng hai hầm chui dưới dạ cầu Điện Biên Phủ để kết nối thông suốt hai con đường Hoàng Sa và Trường Sa.

Có gần 20 cây cầu bắc ngang kênh trong đó 9 cây cầu được đánh số. Các cây cầu còn lại có tên lần lượt là: cầu Lê Văn Sỹ, C. Trần Quang Diệu, C. công Lý, C. Trần Khánh Dư, C. Trần Nguyên Đán, C. Bông.

Cầu Thị Nghè là cây cầu bắc qua rạch Thị Nghè (đoạn gần hạ lưu đổ vào sông Sài Gòn), nối đường Nguyễn Thị Minh Khai thuộc Quận 1 với đường Xô Viết Nghệ Tĩnh thuộc quận Bình Thạnh. Cầu có chiều dài 105,2 mét, rộng 17,6 mét. Có dải phân cách giữa 4 làn xe chạy. Cầu xưa kia do bà Nguyễn Thị Khánh, con gái quan khâm sai Nguyễn Cửu Vân xây để chồng tiện đường qua Sài Gòn làm việc. Chồng bà chỉ là thư ký, không rõ đã đạt đỗ gì, nhưng đương thời gọi là ông Nghè (tức đã đỗ Tiến sĩ), nên nhân dân gọi bà là Bà Nghè. Từ giữa thế kỷ 19, cầu được gọi là cầu Thị Nghè. Đến năm 1970, cầu được xây mới bằng xi măng cốt thép.

Đây là khu vực [hoạt động mại dâm](https://vnexpress.net/cho-tinh-cau-thi-nghe-1975177.html) khét tiếng một thời.

Cầu Thị Nghè 2 bắc qua [kênh Nhiêu Lộc – Thị Nghè](https://vi.wikipedia.org/wiki/K%C3%AAnh_Nhi%C3%AAu_L%E1%BB%99c_%E2%80%93_Th%E1%BB%8B_Ngh%C3%A8 "Kênh Nhiêu Lộc – Thị Nghè"). Nằm trên trục đường [Nguyễn Hữu Cảnh](https://vi.wikipedia.org/wiki/Nguy%E1%BB%85n_H%E1%BB%AFu_C%E1%BA%A3nh "Nguyễn Hữu Cảnh") nối quận 1 với quận Bình Thạnh. Cầu có 10 làn xe và dải phân cách ở giữa chạy 2 chiều.

---

**Cầu Điện Biên Phủ** (tên gọi cũ là cầu Phan Thanh Giản) là một cây cầu bắc qua kênh Nhiêu Lộc – Thị Nghè trên đường Điện Biên Phủ, nối quận 1 với quận Bình Thạnh, Thành phố Hồ Chí Minh.

---

**Cầu Bông** là một cây cầu bắc qua kênh Nhiêu Lộc – Thị Nghè tại Thành phố Hồ Chí Minh, nối đường Đinh Tiên Hoàng thuộc Quận 1 với đường Lê Văn Duyệt thuộc quận Bình Thạnh.

**Cầu Kiệu** là một cây cầu bắc qua kênh Nhiêu Lộc – Thị Nghè tại Thành phố Hồ Chí Minh, nối đường Hai Bà Trưng thuộc Quận 1 và Quận 3 với đường Phan Đình Phùng thuộc quận Phú Nhuận.

Cầu Kiệu là một trong những cây cầu có lịch sử lâu đời nhất tại Thành phố Hồ Chí Minh.

\section{Đ. Nguyễn Văn Trỗi--Nam Kỳ Khởi Nghĩa \& Pasteur \& Huyền Trân Công Chúa}

**Đường Nam Kỳ Khởi Nghĩa**, trước đây là **đường Công Lý** là một tuyến đường tại khu vực trung tâm Thành phố Hồ Chí Minh, nối từ cầu Công Lý (Quận 3) đến đại lộ Võ Văn Kiệt (Quận 1). Tuyến đường này cùng với đường Nguyễn Văn Trỗi trên địa bàn quận Phú Nhuận là trục giao thông quan trọng kết nối Sân bay quốc tế Tân Sơn Nhất với trung tâm thành phố. Nam Kì Khởi Nghĩa khi giao với đường Điện Biên Phủ (tại ngã tư Marie Curie) thì trở thành đường một chiều.

**Đường Pasteur** và đường Nam Kỳ Khởi Nghĩa là hai con đường chạy song song và lưu thông ngược chiều nhau tại khu vực trung tâm Thành phố Hồ Chí Minh. Đường Công Lý chạy về phía hầm Thủ Thiêm, còn đường Pasteur thì chạy ngược lại. Đây cũng là nơi đặt cơ sở đầu tiên của Viện Pasteur Sài Gòn trước khi dời về địa điểm hiện nay.

NKKN: La Vela Saigon Hotel, Trường THPT Marie Curie, hồ con rùa, ĐH kiến trúc, POCPOC beer Garden, Trường THPT Lê Quý Đôn, Công viên 30/4, Tòa án Nhân dân Tp. Hồ Chí Minh

Đây là đường nằm phía sau dinh độc lập. Nằm giữa dinh và CV Tao Đàn. Đường ngắn có chiều dài đúng bằng chiều ngang của dinh. Là đường 1 chiều đi từ Hồng Thập Tự qua Nguyễn Du.

Đường Nguyễn Văn Trỗi là một trong những con đường huyết mạch của quận Phú Nhuận. Đường đi từ cầu Công Lý (nối tiếp Đ. Nam Kỳ Khởi Nghĩa) tới công viên Hoàng Văn Thụ.

Nguyễn Văn Trỗi là liệt sĩ cách mạng, chiến sĩ biệt động nội thành, sinh ngày 1 - 2 - 1940 tại làng Thanh Quít, huyện Điện Bàn, tỉnh Quảng Nam (nay là t. Quảng Nam).

\section{Đ. Lê Duẩn, Đồng Khởi}

Đường Lê Duẩn (**đại lộ Norodom**, **đại lộ Thống Nhất**, đường 30 tháng 4) là một con đường tại trung tâm Quận 1, Thành phố Hồ Chí Minh, chạy từ Dinh Độc Lập đến Thảo Cầm Viên.

Tuyến đường này bắt đầu từ đường Nam Kỳ Khởi Nghĩa trước cổng Dinh Độc Lập, cắt qua các tuyến đường: Pasteur, Phạm Ngọc Thạch – Công trường Công xã Paris, Hai Bà Trưng, Lê Văn Hưu, Mạc Đĩnh Chi, Đinh Tiên Hoàng – Tôn Đức Thắng và kết thúc tại đường Nguyễn Bỉnh Khiêm trước cổng Thảo Cầm Viên.

Năm 1986, Ủy ban nhân dân Thành phố Hồ Chí Minh quyết định đổi tên thành đường Lê Duẩn như hiện nay. Tổng Bí thư Lê Duẩn qua đời vào ngày 10 tháng 7 năm 1986, chính quyền Thành phố Hồ Chí Minh quyết định đặt tên đường để vinh danh ông.

Đường Đồng Khởi, trước đây là **đường Tự Do** (1954 - 1975) và **đường Catinat** (1865 - 1954) là một con đường tại Quận 1, Thành phố Hồ Chí Minh, chạy từ nhà thờ Đức Bà Sài Gòn đến Bến Bạch Đằng. Đây là tuyến phố thương mại sầm uất nổi tiếng của thành phố từ thời Pháp thuộc đến nay với nhiều cửa hàng, khách sạn sang trọng.

Đường Đồng Khởi là tuyến đường một chiều, bắt đầu từ đường Nguyễn Du ngay đối diện Công trường Công xã Paris (trước mặt nhà thờ Đức Bà Sài Gòn), cắt qua các con đường Lý Tự Trọng, Lê Thánh Tôn, Lê Lợi – Công trường Lam Sơn, Nguyễn Thiệp, Đông Du, Mạc Thị Bưởi, Hồ Huấn Nghiệp, Ngô Đức Kế và kết thúc tại đường Tôn Đức Thắng đối diện với công viên Bến Bạch Đằng (bờ sông Sài Gòn).

Trên đường: Tòa nhà The Metropolitan, dọc phố đi bộ Nguyễn Huệ, Mandarin Oriental, tứ giác Eden (thương xá Eden, Union Square), Ho Chi Minh City Opera House, Sheraton Saigon Grand Opera Hotel, Hotel Majestic Saigon (số 1 Đồng Khởi), Hotel Continental Saigon, Café Givral (đóng cửa), Brodard Bakery

**Quảng trường Công xã Paris** là quảng trường nhỏ nằm ở quận 1, giữa nhà thờ Đức Bà và đường Nguyễn Du, bên cạnh là Bưu điện Trung tâm.

Năm 1959, tín đồ Công giáo Rôma dựng tượng Đức Bà Hòa Bình tại đây, từ đó khu đất này còn được gọi **công trường Hòa Bình**. Sau sự kiện 30 tháng 4 năm 1975, chính quyền đổi tên mới theo một sự kiện lịch sử diễn ra vào cuối thế kỷ 19 tại Pháp là Công xã Paris.

\section{Công viên 30/4, CV Tao Đàn}

Công viên 30/4 (Công viên Ba Mươi Tháng Tư) là một công viên ở trung tâm Thành phố Hồ Chí Minh, Việt Nam, thuộc địa phận Quận 1, đối diện Dinh Độc Lập, sau lưng Nhà thờ Đức Bà Sài Gòn.

Công viên gồm bốn ô đất hình chữ nhật do bị chia làm tư bởi đường Lê Duẩn và đường Pasteur chạy xuyên qua, tổng diện tích mặt bằng 3,5 ha, trong đó cây xanh 2,5 ha. Tứ phía công viên là các con đường gồm **Hàn Thuyên**, **Alexandre de Rhodes**, Phạm Ngọc Thạch và Nam Kỳ Khởi Nghĩa. Cặp đôi đường Hàn Thuyên và đường Alexandre de Rhodes như để vinh danh hai nhân vật có công lao phát triển, truyền bá chữ Nôm và chữ Quốc ngữ ở Việt Nam. Hai đường Hàn Thuyên và Alexandre de Rhodes chỉ dài đúng bằng chiều dài công viên.

\section{Đ. Yersin}

Đường Yersin nằm ở phường Cầu Ông Lãnh, quận 1, dài hơn 600 m nối từ đường Võ Văn Kiệt đến Phạm Ngũ Lão.

Alexandre Émile Jean Yersin (1863-1943) là bác sĩ y khoa, nhà vi khuẩn học và nhà thám hiểm người Pháp gốc Thụy Sĩ. Ông khám phá cao nguyên Lâm Viên và vạch ra con đường bộ từ Trung Kỳ sang Cao Miên. Yersin cũng thành lập và là hiệu trưởng đầu tiên của trường y Đông Dương (tiền thân của Đại học Y Hà Nội).

\section{Đ. Hai Bà Trưng}

Đường này đi qua hai quận trung tâm là Quận 1 và Quận 3. Đây là một trong những con đường sầm uất và lâu đời nhất tại Thành phố Hồ Chí Minh.

Đường Hai Bà Trưng dài khoảng 2,97 km, đi từ công trường Mê Linh bên sông Sài Gòn đến đầu cầu Kiệu bắc qua kênh Nhiêu Lộc – Thị Nghè, cắt qua các con đường sầm uất khác tại khu vực trung tâm thành phố như: Lê Thánh Tôn, Lý Tự Trọng, Lê Duẩn, Nguyễn Thị Minh Khai, Nguyễn Đình Chiểu, Điện Biên Phủ, Võ Thị Sáu.

Nhà thờ Tân Định nằm trên đường Hai Bà Trưng là một nhà thờ cổ được xây dựng theo phong cách La Mã với màu hồng đặc trưng.

Nằm hai bên đường: chợ Tân Định, Công Viên Lê Văn Tám, Tổng lãnh sự quán Trung Quốc, hồ con rùa, Tổng Lãnh Sự Quán Pháp, Trường THPT Chuyên Trần Đại Nghĩa

\section{Ngã tư Bảy Hiền \& Bùng binh Lăng Cha Cả}

**Ngã tư Bảy Hiền** là nơi giao nhau của 4 con đường khác nhau: Trường Chinh – Cách Mạng Tháng Tám, Hoàng Văn Thụ - Lý Thường Kiệt. Nằm ngay Bv Thống Nhất, kế bên Bành Văn Trân.\
Ông Bảy Hiền là một điền chủ giàu có thương người.\
Gần ngã tư Bảy Hiền có [nút giao Lý Thường Kiệt](https://maps.app.goo.gl/ytvviTWhDCPyenw36) với đường Lạc Long Quân.

**Lăng Cha Cả** là lăng mộ của Giám mục Bá Đa Lộc (tục gọi là "Cha Cả", tức Pierre Joseph Georges Pigneau de Béhaine). Ngôi mộ xưa là một di tích lịch sử ở Thành phố Hồ Chí Minh. "Lăng Cha Cả" còn được dùng để gọi khu vực gần mộ, nay thuộc địa phận phường 4, quận Tân Bình. Ngôi mộ đã được giải tỏa trong thập niên 1980, còn nay nơi đây là một nút giao thông cùng mức dưới hình thức một vòng xoay giao thông, ở giữa có đặt **quả địa cầu** lớn. Nằm ở góc phía Tây của tam giác Cv. Hoàng Văn Thụ.\
Lăng Cha Cả là nơi giao nhau giữa: Hoàng Văn Thụ (có cầu vượt HVT), Cộng Hòa, Lê Văn Sỹ, Thăng Long, Trần Quốc Hoàn, Bùi Thị Xuân.

\section{Đ. Lý Thường Kiệt}

Vị trí: Đường nằm trên địa bàn quận Tân Bình, các phường 14, 7 quận 10 và chung với quận 11, quận 5, từ ngã tư Bảy Hiền đến đường Hồng Bàng, qua ngã tư Lê Minh Xuân, Nghĩa Phát, ngã ba Tô Hiến Thành, ngã tư 3 tháng 2, Vĩnh Viễn, Hoà Hảo, Tân Phước.

Lịch sử: Đường này trước kia là hai đường nối đuôi nhau. Đoạn từ Hồng Bàng đến giáp ranh quận Tân Bình mang tên Lý Thường Kiệt. Đoạn còn lại là đường Marechal Foch, đến ngày 22-3-1955 đổi là đường Nguyễn Văn Thoại. Ngày 14-8-1975 nhập hai đường làm một lấy tên Lý Thường Kiệt.

Gần ngã tư Bảy Hiền có [nút giao mũi tàu Lý Thường Kiệt](https://maps.app.goo.gl/ytvviTWhDCPyenw36) với đường Lạc Long Quân.

Bv Thống Nhất (số 1 LTK), Bv 1A, chợ Tân Bình, CMC Plaza siêu thị Nguyễn Kim (đã đóng cửa), đại học bách khoa (268 TK), trường đua Phú Thọ, Bv Hùng Vương, Hùng Vương Plaza

\section{Đ. Bắc Hải}

Bắc Hải đồng thời là tên đường và cư xá.

 Đường Bắc Hải ở quận 10, cắt đường Cách Mạng Tháng Tám, Thành Thái và Lý Thường Kiệt với chiều dài hơn 600 mét.

Thời Pháp thuộc, đây là con hẻm của làng Chí Hòa. Năm 1946, Pháp xây dựng cư xá sĩ quan cho quân đội liên hiệp, đường được mở rộng mang tên Quân Sự. Từ năm 1959, cư xá này được gọi là cư xá sĩ quan Chí Hòa.

Mười năm sau, cả cư xá lẫn con đường trên được đổi tên Bắc Hải.

Nhà thiếu nhi Q10, sân khấu C30 Hòa Bình, Đoàn Lô Tô Sài Gòn Tân Thời, trung tâm văn hóa Q10

Cư xá Bắc Hải

Cư xá Bắc Hải được xây dựng trước năm 1975, tên gọi cũ là Cư xá sĩ quan Chí Hòa, sau đổi thành Bắc Hải. Khu cư xá Bắc Hải nằm ở sau công viên Lê Thị Riêng, và ở một thành phố với nhiều khu cư xá nổi tiếng như cư xá Đô thành ở Vườn Chuối, Cư xá Đài ra-đa hack não anh em shipper, thì cư xá Bắc Hải lại nổi tiếng vì đặc điểm riêng của nó nhưng lại là đặc điểm chung của quận 10: ăn và chơi.

Đầu tiên khi vừa đặt chân đến Bắc Hải, đó là bạn phải có kĩ năng đọc google maps, hoặc ít nhất là bạn không mù đường. Tuy rằng đường sá ở đây thông nhau, nhưng 9 nhánh Cửu Long ở Bắc Hải cũng đủ làm cho các bánh bèo mù đường lần đầu đặt chân đến đây chia ra làm hai loại: một loại gọi người thân trợ giúp, loại còn lại sẽ vòng vòng đến khi hết xăng mà vẫn chưa tìm ra được điểm khác nhau giữa đường Bửu Long và đường Cửu Long, chẳng hiểu tại sao giữa quận 10 mà lại có thể có đầy đủ Trường Sơn, Bạch Mã, Ba Vì, Tam Đảo.

\section{Ngã ba ông Tạ}

Ngã ba là điểm giao giữa hai đường Cách Mạng Tháng Tám và Phạm Văn Hai thuộc phường 5 (quận Tân Bình). Cư dân khu vực Ngã ba Ông Tạ đa số người miền Bắc và phần lớn theo đạo Thiên chúa.

\section{Ngã tư Hàng Xanh}

Đây là nút giao thông quan trọng tại cửa ngõ phía Đông Sài Gòn. Ngã tư Hàng Xanh là điểm giao nhau của 2 tuyến đường huyết mạch Điện Biên Phủ và Xô Viết Nghệ Tĩnh.

\section{Vòng xoay công trường Dân Chủ}

Democracy

 Công trường Dân Chủ là một vòng xoay giao thông giữa Quận 3 và Quận 10. Đây là nơi giao nhau của sáu con đường, gồm Cách Mạng Tháng Tám (2 nhánh), Võ Thị Sáu, Lý Chính Thắng, Nguyễn Phúc Nguyên (đi vào ga tàu lửa SG), Ba Tháng Hai và Nguyễn Thượng Hiền. Đây là nơi tiếp giáp của Quận 3 và Quận 10 (to the SW).

 Dưới triều Vua Minh Mạng, xảy ra cuộc nổi dậy Lê Văn Khôi (1833-1835) ở thành Phiên An (thành Gia Định). Triều đình đàn áp thành công cuộc nổi dậy, sát hại 1.831 người cả già trẻ, trai gái ở trong và bên ngoài thành vài dặm, sau đó chôn xác chung một chỗ, gọi bãi đó là **Mả ngụy** hay **Mả biền tru**.

Theo nhà nghiên cứu Nguyễn Đình Đầu, vị trí ngôi mộ tập thể đó nằm ở khoảng gần Công trường Dân Chủ, đường Cách Mạng Tháng Tám và đầu đường Ba Tháng Hai.

Đặc điểm nhận dạng:

- Không có tượng chính giữa
- Có tòa nhà Viettel Tower rất nổi bật
- gần: CAPELLA GALLERY HALL, góc cua hẹp Trường THCS Lê Lợi từ đường một chiều Võ Thị Sáu, Nhà thờ Mai Khôi mái ngói cam, MIA.vn Dân Chủ - Siêu thị vali kéo du lịch, Balo, túi xách, phụ kiện

\section{Đ. Cách mạng tháng 8, Trường Chinh, Cộng Hòa}

CMT8

Trước đây còn gọi là **đường Lê Văn Duyệt**. Bắt đầu từ ngã sáu Phù Đổng Q1, đi qua bùng binh công trường Dân Chủ Q3 và kết thúc ở ngã tư Bảy Hiền. Đường là ranh giới giữa Q3 và Q10 trên đoạn từ công trường Dân Chủ tới CV Lê Thị Riêng.

Trên đường: Phong Vu Computer, Công viên Tao Đàn, Đài tưởng niệm Bồ tát Thích Quảng Đức, Starbucks CMT8

Trường Chinh

CMT8 tới ngã tư Bảy Hiền đổi tên Trường Chinh chạy tới [ngã tư An Sương](https://maps.app.goo.gl/yS8QBv86YTf64acu5).

Đ. Cộng Hòa

Đi Hóc Môn, Củ Chi và tỉnh Tây Ninh

- Pandora City or GO! Trường Chinh nằm ngay mũi tàu giao với Cộng hòa.
- Khu công nghiệp Tân Bình nằm ngay [Cầu Tham Lương](https://maps.app.goo.gl/HDCK7jybf6rD1XVaA).

\section{Cư xá Đô Thành}

96,69% số người lần đầu đến đây phải lấy Amway làm điểm mốc để tới khu cư xá, 3,31% còn lại thì phải đi lên **chợ Vườn Chuối** rồi vòng ngược trở lại vì chạy lố đường.

Vì ở quận 3, nên số biệt thự ở đây chiếm đa số, và gần như là đóng cửa nguyên ngày, nhưng mỗi lần mấy căn đó mở cửa, thì toàn là để cho xế hộp đi ra.

Có thể xem là một cư xá sinh viên – học sinh với một trường thcs, một trường Nhật ngữ, trung tâm luyện thi, viện Goethe, chỗ dạy thêm và n cái nhà trọ.

\section{Bành Văn Trân}

Trước 75 tên là đường **Thánh Mẫu \& Nhà Thờ**

\section{Nguyễn Kiệm}

Cầu vượt Nguyễn Thái Sơn – Nguyễn Kiệm - Hoàng Minh Giám](https://maps.app.goo.gl/MedTbi6nLv3zTZt3A). Nằm ngay công viên Gia Định. Là nơi giau nhau của các con đường: Bạch Đằng, Hồng Hà, Hoàng Minh Giám (hướng về [Bộ Tư lệnh Quân khu 7](https://maps.app.goo.gl/Nfjzo9bbep2CkoFHA)), Nguyễn Thái Sơn (về Thủ Đức), Nguyễn Kiệm, Phạm Văn Đồng

[Cầu vượt ngã sáu Nguyễn Kiệm](https://maps.app.goo.gl/U9JwoDBVLsUJXaBg8). Nằm ngay GO! Gò Vấp

\section{Đ. Nguyễn Thị Minh Khai}

Đường Nguyễn Thị Minh Khai là một con đường tại trung tâm Thành phố Hồ Chí Minh, đi từ ngã sáu Cộng Hòa đến cầu Thị Nghè. Đường dài khoảng 4 km, đi qua hai quận trung tâm của thành phố là Quận 1 và Quận 3. Trong đó, đoạn từ đường CV Tao Đàn tới đường Phùng Khắc Khoan là đường một chiều.

Năm 1955, chính quyền Việt Nam Cộng hòa đổi tên đường Chasseloup-Laubat thành đường **Hồng Thập Tự**, do Hội Hồng Thập Tự và Bộ Y tế khi đó đều đặt trụ sở trên đường này, tại góc phía tây của công viên Tao Đàn (nay là trụ sở Sở Y tế Thành phố Hồ Chí Minh).

Năm 1975, chính quyền Cộng hòa Miền Nam Việt Nam nhập đường Hồng Thập Tự với đường Hùng Vương (phía bên kia cầu Thị Nghè, thuộc địa bàn tỉnh Gia Định) thành đường Xô Viết Nghệ Tĩnh đi từ ngã sáu Cộng Hòa đến kinh Thanh Đa. Tuy nhiên đến năm 1991, Ủy ban nhân dân Thành phố Hồ Chí Minh lại quyết định cắt đoạn đường từ ngã sáu Cộng Hòa đến cầu Thị Nghè (tức đường Hồng Thập Tự cũ) đặt thành đường Nguyễn Thị Minh Khai như hiện nay.

Từ ngã sáu Cộng Hòa tới ngã tư giao với Hai Bà Trưng, đường Hồng Thập Tự là ranh giới của Q1 và Q3.

Nằm dọc trên đường: Trường THPT Lê Quý Đôn, dinh độc lập, nhà sách Minh Khai, nhà sách Cá Chép 2, Bệnh Viện Từ Dũ, Tổng Lãnh Sự Quán Pháp

THPT Nguyễn Thị Minh Khai dạy tiếng Pháp nhiều.

\section{Ngã tư Bốn Xã}

Ngã tư Bốn xã – nay đã trở thành ngã sáu nhưng nhiều người vẫn gọi với cái tên quen thuộc là ngã tư 4 xã thuộc địa phận quận Bình Tân, TP/.HCM. Điểm giao thông này giao giữa các con đường Thoại Ngọc Hầu – Hương Lộ 2 – Lê Văn Quới – Hòa Bình – Bình Long – Phan Anh.

Trước kia đây là ngã tư, nằm gầm 4 xã Phú Thọ Hòa, Tân Thới Hòa, Bình Hưng Hòa và Bình Trị Đông, nhưng nay các xã này đã đổi tên thành quận cho phù hợp hơn. Trước kia, trực thuộc huyện Bình Chánh và quận Tân Bình của Sài Gòn.

Hiện nằm ở ranh giới giữa hai quận Tân Phú và Bình Tân

\section{Đ. Điện Biên Phủ}

Đường đi từ vòng xoay ngã bảy Lý Thái Tổ, Lê Hồng Phong, Ngô Gia Tự (Quận 3) đến tháp đồng hồ Đa Kao, đi qua cầu Điện Biên Phủ rồi tới ngã tư Hàng Xanh rồi tới dưới chân cầu Sài Gòn (quận Bình Thạnh) là kết thúc. Qua cầu Sài Gòn là bắt đầu đường Võ Nguyên Giáp. Đường đi ngang Q1, Q3, và một phần Q10.

Là đường 1 chiều từ ngã bảy tới ngã tư giao với Đinh Tiên Hoàng, gần tháp đồng hồ.

Điện Biên Phủ là trục chính xuyên tâm nội đô TP/.HCM, kết nối các quận 10, 3, 1, Bình Thạnh với thành phố Thủ Đức để ra trục Xa lộ Hà Nội (nay là đường Võ Nguyên Giáp) về Đồng Nai, nhập vào quốc lộ 1 ra các tỉnh miền Trung, miền Bắc.

Nằm trên đường: Tổng Lãnh sự quán Nhật Bản, Trường Đại học Văn Hiến, Bệnh viện Bình Dân (màu xanh blue), Starbucks CMT8 (tòa nhà Thorakao), Bệnh viện Mắt, THPT Nguyễn Thị Minh Khai, THPT Marie Curie, Đại sứ quán Hoàng gia Cam-pu-chia, Công Viên Lê Văn Tám, Tháp Đồng hồ Đa Kao, cổng vào cư xá Đô Thành (383 Điện Biên Phủ, giao với Nguyễn Hiền), Đại Học Quốc Tế Hồng Bàng, ĐH. HUTECH

\section{Tháp đồng hồ Đa Kao}

Hay **Vòng xoay Điện Biên Phủ**, là nút giao nằm ở khu trung tâm TPHCM, được xem như một biểu tượng của thành phố với tháp đồng hồ lâu đời.

Là điểm giao cắt của: Điện Biên Phủ, Nguyễn Bỉnh Khiêm và đường Hoàng Sa, kết nối giao thông giữa cửa ngõ phía Đông với trung tâm thành phố.

\section{Đ. Nguyễn Bỉnh Khiêm}

Đi từ tháp đồng hồ Đa Kao tới cảng Ba Son, lên cầu Thị Nghè 2.

Nhà hàng CCCP Sài Gòn, Chi cục Thuế Quận 1, Gem Center, Bảo tàng Lịch sử , sở thú, Tòa nhà PetroVietnam Tower


\section{Đ. Lý Chính Thắng}

Là đường một chiều.

Vị trí: Đường nằm trên địa quận 3, từ đường Hai Bà Trưng đến công trường Dân Chủ, dài khoáng 1674 mét, qua ngã ba Huỳnh Tịnh Của, ngã tư Nam Kỳ Khởi Nghĩa (La Vela hotel), ngã ba Đoàn Công Bửu, các nga tư Trần Quốc Thảo, Trương Định, Bà Huyện Thanh Quan, Nguyễn Thông.

Trước 1975 tên là **Đ. Yên Đổ**.

Lý Chính Thắng là liệt sĩ cách mạng.

\section{Đ. Hậu Giang \& Tháp Mười}

Đường Hậu Giang là một đường thuộc Quận 6, lưu thông hai chiều kéo dài từ đường Phạm Đình Hổ đến Vòng Xoay Phú Lâm. Đến ngã tư giao với Đ. Phạm Đình Hổ, đường đi thẳng đổi tên thành **đường Tháp Mười**. Đ. Tháp mười sau đó đi được một đoạn ngắn rồi nhập vào Đ. Hải Thượng Lãn Ông.

Cầu Hậu Giang nằm trên đường.

\section{Đ. Hải Thượng Lãn Ông}

Đường Hải Thượng Lãn Ông hay **đường Khổng Tử** là một tuyến đường trên địa bàn Quận 5, Thành phố Hồ Chí Minh. Đây là con đường lớn, có lịch sử lâu đời tại khu vực Chợ Lớn, dọc hai bên đường hiện nay vẫn nhiều ngôi nhà có kiến trúc cổ cùng với nét sinh hoạt truyền thống của cộng đồng người Hoa.

Tuyến đường này bắt đầu từ nút giao với đường Võ Văn Kiệt đoạn giữa cầu Chà Và và cầu Nguyễn Tri Phương, đi về hướng tây và kết thúc tại đường Ngô Nhân Tịnh gần chợ Bình Tây (ngay CV Cửu Long). Nối tiếp là đường Tháp Mười và Đ. Lê Quang Sung.

Nằm trên đường: Miếu ông Bổn

\section{Đ. Châu Văn Liêm}

Đường Châu Văn Liêm (đại lộ Tổng Đốc Phương) là một tuyến đường tại Thành phố Hồ Chí Minh, nối từ đường Hồng Bàng (**ngã tư quỹ tín dụng Chợ Lớn**) đến đường Hải Thượng Lãn Ông trên địa bàn Quận 5. Đây là đoạn đường ngắn nhưng rộng và đẹp.

Tuyến đường này bắt đầu từ nút giao với hai con đường Hồng Bàng và Thuận Kiều, cắt qua các con đường Lão Tử, Nguyễn Trãi, Trần Hưng Đạo và kết thúc tại vòng xoay trước Bưu điện Trung tâm Chợ Lớn (Bưu điện Quận 5), nơi giao với các con đường Hải Thượng Lãn Ông, Nguyễn Thi, Mạc Cửu.

Tại Sài Gòn trước đây có nhiều rạp hát nổi tiếng, là nơi chiếu phim, diễn cải lương, diễn kịch. Trên đường Châu Văn Liêm có ba rạp lớn là Đại Quang, Thủ Đô và Toàn Thắng. Hiện nay chỉ còn rạp Thủ Đô hoạt động tuy nhiên cơ sở vật chất tại đây cũng đã xuống cấp.

\section{Đ. Thuận Kiều}

Từ nơi giao với Hòa Hảo, Trần Quý, Lê Đại Hành đi tới đường Châu Văn Liên.

Đi ngang hông Bv Chợ Rẫy, có rất nhiều hiệu thuốc Tây và dụng cụ y khoa.

\section{Đ. Lê Đại Hành}

Từ trước cổng Bv Chợ Rẫy tới **vòng xoay Lê Đại Hành.**

Lotte mart Lê Đại Hành.

\section{Vòng xoay Phan Đình Phùng}

Là nơi giau nhau giữa các con đường: Hải Thượng Lãn Ông, Châu Văn Liêm, Nguyễn Thi, Mạc Cửu và đường đi lên cầu Chà Và.

Ở gần: Bưu điện Q5

\section{Ngã tư đèn năm ngọn Q5}

Là nơi giau nhau giữa 2 con đường: Trần Hưng Đạo B và Phùng Hưng. Ở ngay đây là **Thương xá Đồng Khánh** và chợ vải **đại quang minh**

\section{Đ. Tản Đà}

Cuối đường Tản Đà nơi nhập vào đường Hồng Bàng tọa lạc nhiều địa điểm nổi bật: Đại Học Y Dược TP, Hùng Vương Plaza, THPT Hùng Vương.

Tòa Hành chánh Chợ Lớn (ảnh trái), nằm trên khu đất ngày nay là đại học Y được TP HCM, mặt chính tòa nhà nhìn thẳng ra đường Jaccaréo (ngày nay là đường Tản Đà). Tòa nhà được người dân địa phương gọi là **Dinh xã Tây**, do quan trưởng xã đều là người Pháp. Khi tòa nhà bị phá bỏ, ở gần khu đất công trình này người ta dựng **chợ Xã Tây** vào năm 1925, còn tồn tại đến ngày nay.

\section{Đ. Ba tháng Hai}

Đường từ cầu vượt Cây Gõ tới bùng binh công trường Dân Chủ.

Đặt tên theo Ngày thành lập Đảng Cộng sản Việt Nam 3-2-1930.

Cầu vượt 3 tháng 2 cắt ngang giao lộ giữa đường Nguyễn Tri Phương (S) và Thành Thái (N) với đường 3/2 và đường Lý Thái Tổ (dẫn về ngã 7 Lý Thái Tổ). Muốn đi vào 3 con đường kia thì đi phía dưới cầu vượt. Gần: Satramart - Siêu thị Sài Gòn

Nằm trên đường: Vạn Hạnh Mall, Bệnh viện Nhi đồng 1, Việt Nam Quốc Tự

\section{Đ. Thành Thái - Nguyễn Tri Phương}

Bắt đầu từ cầu vượt 3/2 (nối tiếp Đ. Thành Thái), đi qua ngã sáu Nguyễn Tri Phương và kết thúc ở cầu Nguyễn Tri Phương qua quận 8.

Nằm trên đường: Chè Thái Ý Phương, Văn phòng Kinh tế và Văn hóa Đài Bắc tại TP/.HCM, Trà Sữa Phượng Hoàng, Đại Học Kinh Tế TP/.HCM (UEH) CS B, Công viên Văn Lang

\section{Đ. Hồng Bàng}

Bắt đầu từ **vòng xoay Phú Lâm** (nối tiếp Kinh Dương Vương). Khi đến đoạn giao với đường Ngô Quyền tại Bệnh Viện Phạm Ngọc Thạch thì tách thành 3 đường: Ngô Gia Tự, Hùng Vương, An Dương Vương.

Nằm 2 bên đường: cầu vượt Cây Gõ, Chùa Tuyền Lâm, THE GARDEN MALL BULDING, Hung Vuong Plaza, Đại Học Y Dược, BV đại học Y dược, Bệnh Viện Phạm Ngọc Thạch

**Đ. An Dương Vương** Là hướng đi thẳng của Hồng Bàng đổ vào Nguyễn Văn Cừ.

\section{Cầu vượt Cây Gõ}

**Nút giao thông Cây Gõ** (hay **Cầu vượt Cây Gõ**) là nút giao thông nằm trên địa bàn Quận 6 và Quận 11, Thành phố Hồ Chí Minh. Đây là nơi giao nhau giữa các tuyến đường Hồng Bàng, Ba Tháng Hai và Minh Phụng.

Vị trí: Nhà sách Nguyễn Văn Cừ, Chùa Sùng Đức, Chùa Huê Lâm

\section{Ngã sáu Nguyễn Tri Phương}

Ngã sáu Nguyễn Tri Phương (tên cũ: Ngã sáu Minh Mạng, **Ngã sáu Chợ Lớn**) là một vòng xoay giao thông ở nơi tiếp giáp giữa Quận 5 và Quận 10 Thành phố Hồ Chí Minh, Việt Nam.

NOTE: Ngã sáu Sài Gòn là bùng binh Phù Đổng.

Ngã sáu là nơi giao nhau của ba đường lớn là Nguyễn Tri Phương, Ngô Gia Tự và Nguyễn Chí Thanh, tạo nên sáu ngã rẽ, nằm ở ranh giới Quận 5 với Quận 10. Nơi này còn có tên cũ là Ngã sáu Minh Mạng vì đường Ngô Gia Tự trước năm 1975 có tên là đường Minh Mạng. Phần lõi (đảo giao thông) có một cột cao kiểu thức cột Corinth, toàn bộ quét màu trắng. Bệ cột này trước 1975 có khắc: "Kỷ niệm Quốc khánh 1.11.1966". Trên đỉnh cột có bức tượng An Dương Vương tay cầm nỏ. Công trình do Binh chủng Công binh Việt Nam Cộng hòa thực hiện vì họ xem An Dương Vương là "thánh tổ" của lực lượng.

\section{Đ. Nguyễn Chí Thanh}

Chạy từ đường 3/2 tới mũi tàu Cv Hòa Bình giao tới đường Hùng Vương.

Ngang mặt tiền Bv Chợ Rẫy \& Bv RHM TW\
chợ Hà Tôn Quyền ẩm thực người tàu

\section{Đ. Ngô Gia Tự}

Đi từ điểm cuối Hồng Bàng (ngay công viên Văn Lang), qua ngã sáu Nguyễn Tri Phương và kết thúc ở ngã 7 Lý Thái Tổ.

\section{Đ. Tên Lửa}

Đường Tên Lửa từ đường Kinh Dương Vương đến liên tỉnh lộ 10. Đơn vị tên lửa quân đội giải phóng đóng doanh trại tại đây.

\section{Đ. Kinh Dương Vương}

Vòng xoay An Lạc -> Bùng binh Phú Lâm.

