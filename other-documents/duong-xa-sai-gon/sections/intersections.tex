\chapter{Intersection - Vòng xoay, ngã tư, nút giao}

\section{Ngã ba Dầu Giây}

Ngã ba Dầu Giây thuộc huyện Thống Nhất, tỉnh Đồng Nai đây là đoạn cuối đường cao tốc TP.HCM - Long Thành - Dầu Giây tỏa hướng ra Quốc lộ 20 (đi Đà Lạt) và Quốc lộ 1.

Ngã ba nằm ngay khúc giao với QL20, gần đó 2KM là đường kết nối với cao tốc TP.HCM - Long Thành - Dầu Giây.

\section{Ngã sáu Cộng Hòa}

[Ngã sáu Cộng Hòa](https://vi.wikipedia.org/wiki/Ng%C3%A3_s%C3%A1u_C%E1%BB%99ng_H%C3%B2a) (tên cũ: **Công trường Cộng Hòa**) là một vòng xoay giao thông nằm ở nơi giao nhau của sáu con đường, gồm Nguyễn Văn Cừ, Phạm Viết Chánh, Nguyễn Thị Minh Khai, Lý Thái Tổ, Hùng Vương và Trần Phú. Đây là nơi tiếp giáp của bốn quận.

Sở dĩ có tên Ngã sáu Cộng Hòa vì đường Nguyễn Văn Cừ trước 1975 là **Đường Cộng Hòa**. Hồi đó chính giữa bùng binh có tượng đài Cảnh Sát Quốc Gia và đi ngang qua đây còn có đường sắt Sài Gòn - Mỹ Tho. Một góc bùng binh là **công viên Âu Lạc** nằm giữa mũi tàu đường Hùng Vương và đường Trần Phú. Đối diện là góc vườn hoa nhỏ kẹp giữa hai con đường Nguyễn Thị Minh Khai và Phạm Viết Chánh.

Thời Việt Nam Cộng hòa, ở giữa công trường có đặt bức tượng khắc họa hình ảnh một viên Cảnh sát Quốc gia.

Ngã sáu Cộng Hòa là giao điểm của 4 quận trung tâm Tp.HCM: Quận 1, Quận 3, Quận 5, Quận 10 nên suốt ngày đêm xe cộ qua lại nhộn nhịp. Gần đó có các trường học lớn: ĐH Khoa học tự nhiên, ĐH Sư phạm Tp.HCM, ĐH Sài Gòn, Trường THPT Chuyên Lê Hồng Phong (xưa là Trường Pétrus Ký) và trung tâm mua sắm Now Zone, Nhà khách Chính phủ ở Quận 10, cùng nhiều nhà hàng, tiệm buôn bán khác nên khu vực này rất sầm uất.

Đặc điểm nhận dạng:

- Không có tượng chính giữa, hình ngôi sao nhìn từ trên xuống.
- Khác với:
- Ở gần: NOWZONE Mall, chuyên LHP, Cơm Tấm Phúc Lộc Thọ, nhà khách chính phủ, Nhà Hàng Tiệc Cưới Phúc An Khang, Công Viên Phong Châu (nhỏ xíu)
- Chợ Bàn Cờ, một [khu đất](https://maps.app.goo.gl/6RZRNTBynU9wmwvZ9) rộng bỏ hoang trước đây của [chú hỏa](https://vi.wikipedia.org/wiki/Ch%C3%BA_H%E1%BB%8Fa).

\section{Ngã sáu Phù Đổng}

Ngã sáu Phù Đổng (còn gọi là Ngã sáu Phù Đổng Thiên Vương hoặc **Ngã sáu Sài Gòn**) là một vòng xoay giao thông ở Quận 1, Thành phố Hồ Chí Minh, Việt Nam. Đây là giao điểm của sáu con đường, có lưu lượng xe cộ qua lại đông đúc. Ở giữa có một bức tượng Thánh Gióng được dựng vào năm 1966.

Là giao điểm của sáu con đường gồm Cách Mạng Tháng Tám, Lý Tự Trọng, Phạm Hồng Thái, Nguyễn Thị Nghĩa, Nguyễn Trãi và Lê Thị Riêng, nằm trọn trong địa phận phường Bến Thành, Quận 1

\section{Ngã năm Cống Quỳnh}

Vòng xoay Ngã 5 Thái Bình xưa với Nhà Thờ Huyện Sĩ, bên trái là rạp Cinéma Khải Hoàn góc Cống Quỳnh-Võ Tánh-Phạm Ngũ Lão, bên phải là chợ Thái Bình và đối diện chợ Thái Bình là "Hội Dục Anh", đường Võ Tánh này ngày nay là đường Nguyễn Trãi bắt đầu từ đường Cộng Hòa.

\section{Ngã ba chú Ía}

Ngã ba Chú Ía, mà nay đã mở rộng thành ngã sáu Nguyễn Thái Sơn (hay vòng xoay NTS), nằm gần công viên Gia Định. Đây là điểm giao giữa các tuyến Nguyễn Kiệm, Nguyễn Thái Sơn, Hoàng Minh Giám, Phạm Văn Đồng.

Theo một nhà nghiên cứu, trước 1975, khu vực này có một người Hoa tên Hía làm nghề thủ công và có cửa hàng Bách hoá lớn nên người Sài Gòn gọi khu vực này thành ngã ba Chú Hía. Qua năm tháng, phát âm này dần biến mất chỉ còn "Chú Ía" cho đến nay.

\section{Ngã tư Linh Xuân}

Là nút giao thông lớn tại khu vực phía Đông TP.HCM

Ngã tư Linh Xuân là nút giao Quốc lộ 1 - Quốc lộ 1k - Đại lộ Phạm Văn Đồng, TP Thủ Đức, đây được đánh giá là nút giao nhỏ hẹp nằm ở cửa ngõ quan trọng nối TP.HCM - Đồng Nai.

\section{Ngã bảy Lý Thái Tổ}

[Ngã bảy Lý Thái Tổ](https://vi.wikipedia.org/wiki/Ng%C3%A3_b%E1%BA%A3y_L%C3%BD_Th%C3%A1i_T%E1%BB%95), hay được gọi tắt là **Ngã Bảy**, là một vòng xoay giao thông ở nơi tiếp giáp giữa Quận 10 và Quận 3, Thành phố Hồ Chí Minh, Việt Nam.

Đây là nơi giao nhau của bốn con đường, gồm Lý Thái Tổ (2 nhánh), Điện Biên Phủ, Lê Hồng Phong (2 nhánh) và Ngô Gia Tự, tạo ra sáu ngã rẽ. Tuy nhiên, do tính cả nhánh nhỏ của hẻm 384 đường Lý Thái Tổ (hẻm chợ Phường 10, Quận 10 hay còn gọi là **Chợ Chuồng Bò**) ở mũi tàu giao nhau giữa đường Lê Hồng Phong và đường Lý Thái Tổ, nên tạo thành ngã bảy.

Trước năm 1975, tại đây đặt một tượng đài ba binh sĩ Biệt động quân Việt Nam Cộng hòa, đến đầu tháng 5 năm 1975 thì bị đám đông đập bỏ trong phong trào xóa bỏ tàn tích của chính thể cũ vừa sụp đổ. Về sau, người ta cho dựng một tượng đài khác có nội dung ca ngợi cuộc biểu tình 1 tháng 5 năm 1966 của 40.000 công nhân, người lao động và các tầng lớp thị dân thành phố Sài Gòn đòi Mỹ rút quân, chấm dứt chiến tranh.

Đặc điểm nhận dạng: Cây xăng ngã bảy Lý Thái Tổ; có bức tượng góc cạnh, nhiều đầu nhìn ra nhiều hướng; Sukiya - Cơm Bò Hầm \& Mì Ramen

\section{Bùng binh Cây Liễu}

Giao lộ Nguyễn Huệ – Lê Lợi, thường được biết đến với các tên gọi Bồn Kèn hay **Bùng binh Cây Liễu**, là một vòng xoay giao thông nơi giao nhau giữa hai con đường Nguyễn Huệ và Lê Lợi tại Quận 1, Thành phố Hồ Chí Minh. Đây được xem là vòng xoay đầu tiên của Sài Gòn xưa và của Việt Nam.

Bồn Kèn là tên người dân thường gọi giao lộ này vào thời Pháp thuộc, được cho là xuất phát từ việc vào mỗi chiều thứ bảy, có nhiều người lính đến đây chơi nhạc Tây. Còn Bùng binh Cây Liễu là cách gọi về sau, khi nơi đây là vòng xoay có hàng cây liễu được trồng xung quanh đài phun nước.

\section{Ngã ba Cát Lái, nút giao thông An Phú}

Ngã ba Cát Lái

[Nút giao thông ngã 3 Cát Lái](https://maps.app.goo.gl/WJrH5p5Ari7nFY2S6) (TP Thủ Đức) nằm ở điểm đầu của đại lộ Đông Tây kết nối với Xa lộ Hà Nội. Đây là tuyến đường quan trọng bậc nhất khu vực cửa ngõ phía đông nối TP\.HCM với các tỉnh Đông Nam bộ và miền Bắc.

Tạo điều kiện cho dòng xe đầu kéo, xe tải nặng lưu thông giữa xa lộ Hà Nội và cảng Cát Lái thuận lợi, nhanh chóng. Từ đó có tên gọi.

Nằm cách đó không xa về phía Tây Nam là nút giao thông An Phú.

Nút giao thông An Phú

Còn gọi là **vòng xoay An Phú** nằm trên địa bàn phường An Phú, Q2 [here](https://maps.app.goo.gl/2uqJcGBaQrodeg7LA). Nút giao thông này nằm gần ngã ba Cát Lái về phía Nam. Hiện này còn đang thi công chưa hoàn thành.

Là nơi giao nhau giữa: Mai Chí Thọ (+ đường vào Đồng Văn Cống xuống cảng Cát Lái), Lương Định Của và đường dẫn vào cao tốc CT01.

\section{Vòng xoay Phú Lâm}

Là nơi giao nhau giữa 5 con đường: Nguyễn Văn Luông, Kinh Dương Vương, Bà Hom, Tân Hòa Đông, Hồng Bàng.

Đi lên **cầu Ông Buông** (cầu đôi Phú Lâm).

Đặc điểm: Co/.opmart Phú Lâm, Anh ngữ ILA - Phú Lâm

\section{Vòng xoay Mũi Tàu}

 Là nơi giao nhau giữa các tuyến đường: Kinh Dương Vương, An Dương Vương, Hậu Giang

Nằm ở một đỉnh của CV Phú Lâm.

\section{Vòng xoay An Lạc, cầu vượt An Lạc}

Vòng xoay An Lạc là nơi giao nhau giữa QL1A và Kinh Dương Vương [here](https://maps.app.goo.gl/PASAthiao5aYvSQG8) đi vào vùng nội thành Sài Gòn. Đi qua [cầu An Lạc](https://maps.app.goo.gl/uWaQnxXk556UNVx7A) sẽ vào đường Kinh Dương Vương.

Gần đó có [cầu vượt An Lạc](https://maps.app.goo.gl/Qf2kWPsG53LUdkK87) giữa đại lộ Võ Văn Kiệt và QL1A

Gần đó là GO! An Lạc, nhà tang lễ thành phố.

