%%%%%%%%%%%%%%%%%%%%%%%%%%%%%%%%%%%%%%%%%
% "ModernCV" CV and Cover Letter
% LaTeX Template
% Version 1.3 (29/10/16)
%
% This template has been downloaded from:
% http://www.LaTeXTemplates.com
%
% Original author:
% Xavier Danaux (xdanaux@gmail.com) with modifications by:
% Vel (vel@latextemplates.com)
%
% License:
% CC BY-NC-SA 3.0 (http://creativecommons.org/licenses/by-nc-sa/3.0/)
%
% Important note:
% This template requires the moderncv.cls and .sty files to be in the same 
% directory as this .tex file. These files provide the resume style and themes 
% used for structuring the document.
%
%%%%%%%%%%%%%%%%%%%%%%%%%%%%%%%%%%%%%%%%%

%----------------------------------------------------------------------------------------
%	PACKAGES AND OTHER DOCUMENT CONFIGURATIONS
%----------------------------------------------------------------------------------------

\documentclass[11pt,a4paper,sans]{moderncv} % Font sizes: 10, 11, or 12; paper sizes: a4paper, letterpaper, a5paper, legalpaper, executivepaper or landscape; font families: sans or roman

\usepackage[vietnamese]{babel}
% \usepackage[english]{babel}  % FIXME: using spanish breaks moderncv

\moderncvstyle{classic} % CV theme - options include: 'casual' (default), 'classic', 'oldstyle' and 'banking'
\moderncvcolor{blue} % CV color - options include: 'blue' (default), 'orange', 'green', 'red', 'purple', 'grey' and 'black'

\usepackage{lipsum} % Used for inserting dummy 'Lorem ipsum' text into the template

\usepackage[scale=0.75]{geometry} % Reduce document margins
%\setlength{\hintscolumnwidth}{3cm} % Uncomment to change the width of the dates column
%\setlength{\makecvtitlenamewidth}{10cm} % For the 'classic' style, uncomment to adjust the width of the space allocated to your name

%----------------------------------------------------------------------------------------
%	NAME AND CONTACT INFORMATION SECTION
%----------------------------------------------------------------------------------------

\firstname{Phạm} % Your first name
\familyname{An Hào} % Your last name

% All information in this block is optional, comment out any lines you don't need
\title{Web Developer}
\address{Tân Bình Distric}{Ho Chi Minh City}
% \mobile{+84~83~530~1183}
\phone{+84~83~530~1183}
% \fax{(000) 111 1113}
\email{anhaophamx@gmail.com}
% \social[github]{tgoldenphoenix}
% \born{4 July 1776}
\homepage{tgoldenphoenix.github.io/portfolio\_generator/}{Personal Website} % The first argument is the url for the clickable link, the second argument is the url displayed in the template - this allows special characters to be displayed such as the tilde in this example
\extrainfo{DOB: 25/12/2002}
% \extrainfo{Github: }
\photo[70pt][0.4pt]{pictures/profile} % The first bracket is the picture height, the second is the thickness of the frame around the picture (0pt for no frame)
% \quote{"A witty and playful quotation" - John Smith}

%----------------------------------------------------------------------------------------

\begin{document}

\makecvtitle

\section{Education \& Work Experience}

\subsection{High School}

\cventry{2017--2020}{Student}{\href{https://www.facebook.com/thptchuyenbentre/?locale=vi_VN}{BenTre High School for Gifted Students}}{Major in Biology}{}{}

%------------------------------------------------

% \cventry{2011--2012}{Summer Intern}{\textsc{Lehman Brothers}}{Los Angeles}{}{Rated "truly distinctive" for Analytical Skills and Teamwork.}

%------------------------------------------------

\subsection{Higher Education}

\cventry{2020--2022}{General Doctor Undergraduate}{\href{https://ctump.edu.vn/}{Can Tho University of Medicine and Pharmacy}}{}{}{Back in high school, my only concern was to pass the exams with good grades and follow the goal set by my parents of becoming a doctor. During the COVID-19 lockdown time, I suddenly had more time at home and began to ask myself if I really want to pursue this career for the next 6-8 years. In the end, after much contemplation and consulting with my family, classmates and teachers, I decided to take my leave of absence from school at the end of the 2\textsuperscript{nd} year.}

\cventry{2022--2023}{}{}{}{}{Spent some time finding myself. This was a courageous endeavour that didn't have a title. It was quite important to my overall development though so I'm adding it to my CV. Also it explains the gap in my CV.
\newline{}\newline{}
Detailed achievements:
\begin{itemize}
  \item Learned programming on freeCodeCamp, \textsc{CS50}.
  \begin{itemize}
    \item I also stumbled upon the FOSS community and developed a deep appreciation for it. Today, I use Neovim to write both my \LaTeX math notes and other notes that use the markdown syntax. For my front-end projects though, I still use the vim emulation extension in VSCode. You can check out \href{https://github.com/tgoldenphoenix/my-dotfiles}{\underline{my dotfiles repo on Github by clicking here}}.
  \end{itemize}
  \item Refresh my memory about Math since it was my most favourite subject in High School even though my major was biology.
  \begin{itemize}
    \item I discovered \LaTeX — which, by the way, is what this \textit{curriculum vitae} is written in. I personally think it look gorgeous! I also use \LaTeX to write my math notes. If interested, you can find the source code on my Github.
  \end{itemize}
  \item Started learning Chinese characters. It is worth mentioning that I do not specifically choose to learn a language such as Chinese Mandarin but rather focusing on acquainting myself with the many different characters.
  \begin{itemize}
    \item In the past, Chinese character (chữ Hán) along with "Chữ Nôm"\ was widely used in Vietnam both in official documents and literature. Nowadays, Chinese character is still being used in China, Japan, Hong Kong, Taiwan, Singapore and many other East Asia countries.
    \item Learning Chinese characters therefore gives me a deeper appreciation regarding the East Asian culture sphere as well as open many opportunities in the future.
    % \begin{itemize}
    %   \item Word not sending the correct data to printer
    %   \item Windows trying to print in letter format
    % \end{itemize}
    \item As of now, I can recognize around 2000 Chinese characters and be able to understand basic words and sentences in Mandarin, Cantonese, Japanese and some Vietnamese scripts of old. I still, however, have much to learn.
  \end{itemize}
  % \item Broke the office record for number of kitten pictures in cubicle
\end{itemize}}

\cventry{2023--2025}{Student}{FPT Aptech HCM}{Ho Chi Minh City}{}{I decided to take learning web development seriously. During my two years at FPT Aptech, I found myself in a very nice environment, compared to self-study at home, with enthusiast lecturers and friendly classmates. They had taught me a lot and also gave me the motivation to keep following this path.}

%----------------------------------------------------------------------------------------
%	PROJECTS SECTION
%----------------------------------------------------------------------------------------

\section{Projects}

\cventry{}{Trello UI clone with ReactJS}{}{}{}{Implementing beautiful layout and the Drag-drop feature for task boards and the cards inside them. \href{https://github.com/tgoldenphoenix/trello-clone-web}{\underline{Github repo}, \href{https://youtu.be/UEWHVnhfnT0?si=EsJmOMBYn0g3UnCj}{\underline{video demo}} }
\newline{}\newline{}
  Detailed technologies:
  \begin{itemize}
    \item ReactJS and Material UI for the layout
    \item DND kit for Drag-drop boards \& cards
    \item Redux to store Board data
    \item ESLint, Yarn, Node Version Manager, Vite
  \end{itemize}
}

\section{Skills}

\cvitem{Front End}{\textsc{html5}, \textsc{CSS3(SASS)}, JavaScript, ReactJS, Material UI, TailwindCSS, Redux}
\cvitem{Back End}{NodeJS (Express), RESTful API }
\cvitem{DevOps}{Linux, Git, Github, bash, Neovim, Vite, Tmux}
% \cvitem{Other}{\LaTeX}

\section{Languages}

\cvitemwithcomment{Vietnamese}{Mothertongue}{}
\cvitemwithcomment{English}{FLuent}{Professional Working Proficiency}
\cvitemwithcomment{Chinese}{Basic}{Basic words and phrases only}

\section{Interests}

\renewcommand{\listitemsymbol}{-~} % Changes the symbol used for lists

\cvlistdoubleitem{Math}{\LaTeX}
\cvlistdoubleitem{Chinese character}{Reading fiction}
\cvlistitem{Cycling}

\end{document}
